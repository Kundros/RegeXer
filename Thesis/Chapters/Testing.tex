\chapter{Zhodnocení a testování výsledků}\label{sec:Testing}

Pro testování při vývoji, jsem používal technologii Jest, pro psaní vlastních testů.
To hlavně kvůli udržitelnosti kódu a včasnému zachycení chyb při vývoji.
Testy jsem používal pouze pro část aplikace, která používá hodně algoritmického kódu.
Konkrétně při vývoji knihovny pro práci s regulárními výrazy \ref{sec:Implementation1}.
Testy jsem psal postupně podle případů, které mě napadly, že by mohly rozbít fungování aplikace, či k nesprávným výsledkům.
Tento přístup byl samozřejmě nedostačující a tak při testování samotné vizualizace, se mi povedlo najít několik případů špatných výsledků. 
Ty jsem následně přidal do samotných testů.

Testování proběhlo na počítači s 6 jádrovým procesorem a 12 vlákny, AMD Ryzen\texttrademark ~5 5600X.
Dostupná paměť na použitém stroji, je 32GB RAM.

\begin{table}[!h]
	\centering
    \begin{tabular}{ |c||c|c|c|c|c|c|c|c| }
        \cline{2-9}
        \multicolumn{1}{c|}{} & \multicolumn{8}{c|}{\textbf{Regulární výraz a testovací řetězec}} \\
        \multicolumn{1}{c|}{} & \multicolumn{8}{c|}{Čas v ms} \\
        \hline
        \textbf{Testovací subjekt} & \ref{itm:TD_1a} & \ref{itm:TD_2a} & \ref{itm:TD_3a} & \ref{itm:TD_4a} & \ref{itm:TD_5a} & \ref{itm:TD_6a} & \ref{itm:TD_7a} & \ref{itm:TD_7b}  \\
        \hlineB{3}
        Regexer (vlastní knihovna) & 1.936 & 0.672 & 1.061 & 34.205 & 1.004 & 1.556 & 2.382 & 116.117 \\
        \hline
        RegExp & 0.039 & 0.023 & 0.031 & 0.107 & 0.051 & 0.045 & 0.053 & 0.029 \\
        \hline
        match & 0.037 & 0.024 & 0.028 & 0.034 & 0.032 & 0.035 & 0.035 & 0.025 \\
        \hline\hline
        \rule{0pt}{14pt} \textbf{Nalezený výsledek} & \textcolor{OliveGreen}{\Checkmark} & \textcolor{OliveGreen}{\Checkmark} & \textcolor{OliveGreen}{\Checkmark} & \textcolor{Red}{\XSolid} & \textcolor{OliveGreen}{\Checkmark} & \textcolor{OliveGreen}{\Checkmark} & \textcolor{OliveGreen}{\Checkmark} & \textcolor{Red}{\XSolid} \\
        \hline
        Násobné zpomalení \textit{Regexeru} & 49x & 29x & 34x & 320x & 20x & 35x & 45x & 4000x \\
        \hline
    \end{tabular}
	\caption{Výsledky testování výkonu zpracování regulárních výrazů}
	\label{tab:DebuggerUI}
\end{table}

V tabulce \ref{tab:DebuggerUI} se nachází testovací výsledky časů zpracovávání.
\textbf{Regulární výrazy a testované řetězce} vychází z přílohy \ref{sec:TestingData}. 
Každý testovací subjekt je označený číslem, které značí regulární výraz a písmenem, které signalizuje vybraný testovací řetězec z listu pod výrazem.
Nalezený výsledek, značí zda testovací řetězec byl nalezen, či nikoliv. 
Násobné zpomalení \textit{Regexeru}, je porovnání oproti JavaScriptové implementaci \textit{RegExp}.

Z výsledků je patrné, že vlastní implementace je značně pomalejší a to hlavně v případě nenalezeného výsledku.
Zpomalení i ve větší míře je očekávané, jelikož se jedná o implementaci psanou v jazyce TS resp. JS.
Také vlastní implementace, má za úkol vytvářet strukturu historie zpracování, což standardní implementace neřeší.
Velké zpomalení při neúspěchu mohlo nastat tak, že každé zpracování probíhá pomocí NKA, ale dnešní jazyky často implementují hybridní variantu DKA s NKA.
DKA v tomto případě má velkou výhodu, jelikož se nemusí vykonávat tolik operací.

Výkonově je tedy vlastní implementace \textit{Regexer} pomalejší, ale pro účely této aplikace je tento výkon dostačující.
Pro zlepšení výkonu, by bylo možné přepsat tuto část aplikace do více nízkoúrovňového jazyka, jako je například \textit{C++}, nebo \textit{Rust}.
