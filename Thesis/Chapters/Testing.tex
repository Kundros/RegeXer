\chapter{Zhodnocení a testování výsledků}\label{sec:Testing}

Pro testování při vývoji, jsem používal technologii Jest, pro psaní vlastních testů.
To hlavně kvůli udržitelnosti kódu a včasnému zachycení chyb při vývoji.
Testy jsem používal pouze pro část aplikace, která používá hodně algoritmického kódu.
Konkrétně při vývoji knihovny pro práci s regulárními výrazy \ref{sec:Implementation1}.
Testy jsem psal postupně podle případů, které mě napadly, že by mohly rozbít fungování aplikace, či k nesprávným výsledkům.
Tento přístup byl samozřejmě nedostačující a tak při testování samotné vizualizace, se mi povedlo najít několik případů, které jsem následně přidal do samotných testů.