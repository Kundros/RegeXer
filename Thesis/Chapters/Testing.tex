\chapter{Zhodnocení a testování výsledků}\label{sec:Testing}

Pro testování při vývoji, jsem používal technologii Jest, pro psaní vlastních testů.
To hlavně kvůli udržitelnosti kódu a včasnému zachycení chyb při vývoji.
Testy jsem používal pouze pro část aplikace, která používá hodně algoritmického kódu.
Konkrétně při vývoji knihovny pro práci s regulárními výrazy \ref{sec:Implementation1}.
Testy jsem psal postupně podle případů, které mě napadly, že by mohly rozbít fungování aplikace, či k nesprávným výsledkům.
Tento přístup byl samozřejmě nedostačující a tak při testování samotné vizualizace, se mi povedlo najít několik případů špatných výsledků. 
Ty jsem následně přidal do samotných testů.

\begin{table}[!h]
	\centering
    \begin{tabular}{ |c||c|c|c|c|c|c|c|c| }
        \hline
        Testovací subjekt & 1 & 2 & 3 & 4 & 5 & 6 & 7 & 8  \\
        \hline
        Regexer (vlastní knihovna) & 1.936 & 0.672 & 1.061 & 34.205 & 1.004 & 1.556 & 2.382 & 116.117 \\
        \hline
        RegExp & 0.039 & 0.023 & 0.031 & 0.107 & 0.051 & 0.045 & 0.053 & 0.029 \\
        \hline
        match & 0.037 & 0.024 & 0.028 & 0.034 & 0.032 & 0.035 & 0.035 & 0.025 \\
        \hline\hline
        Nalezený výsledek & \Checkmark & \Checkmark & \Checkmark & \XSolid & \Checkmark & \Checkmark & \Checkmark & \XSolid \\
        \hline
        Násobné zpomalení Regexeru & 49x & 29x & 34x & 320x & 20x & 35x & 45x & 4000x \\
        \hline
    \end{tabular}
	\caption{Testovací data výkonu aplikace}
	\label{tab:DebuggerUI}
\end{table}