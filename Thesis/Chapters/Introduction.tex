\chapter{Úvod}\label{sec:Introduction}

Vyhledávání v textu patří mezi základní problémy, se kterými se velice pravděpodobně potká skoro každý programátor. 
Tento problém se dá řešit mnoha způsoby, avšak ne všechna řešení lze použít univerzálně a každý ze způsobů má své výhody a nevýhody.
Jedním ze přístupů je využití \textbf{regulárních výrazů}. 
Jedná se o sadu znaků, které nám umožňují nadefinovat výraz a ten je následně převedený do nějaké struktury. 
Nejčastěji jejich výsledná forma je nějaký \textbf{konečný automat}, které jsou blíže vysvětleny v kapitole \ref{sec:Principle}, sekci \ref{sec:FiniteAutomaton}.
Téměř každý dnešní programovací jazyk je obsahuje, ale jejich implementace se mohou lišit.

Cílem této práce je na implementovat nástroj, který bude schopný procházet regulární výrazy a následně vizualizovat tyto průchody, jako rozšíření do zvoleného vývojového prostředí.

Při vývoji programů, je programátor často obeznámen s regulárními výrazy, jedná se totiž o poměrně rychlé řešení pro vyhledávání v textu. 
Můžeme se s nimi setkat, v podstatě skoro ve všech částech softwaru\footnote{počítačový program, aplikace}, např. validace formulářů, vyhledávání v textu nebo třeba v příkazovém řádku.
Avšak tyto výrazy se brzy mohou stát hůře čitelnými, jelikož neumožňují v podstatě skoro žádné formátování\footnote{upravení vzhledu, tvaru}. 
Taktéž mohou být pro mnoho lidí matoucí, či nepřehledné.
Z tohoto důvodu se hodí mít nástroj, který potencionálně usnadní práci programátorům, tak aby si mohli zobrazit průchod zadaným výrazem.
Dále pro lidi, kteří například vidí tyto výrazy poprvé v životě může být snazší jim porozumět, je-li jim ukázáno jak fungují v jednotlivých krocích.
Sice již existují řešení tohoto problému a to v různých formách \cite{Dib, Regexper, RegExr}, ale pro zvolené vývojové prostředí mnoho přístupů neexistuje.
Tato situace je motivací, zabývat se problémem do hloubky a pokusit se nabídnout originální řešení v daném směru, které by mohlo být přínosem pro ostatní.

Implementace těchto výrazů bývá nejčastěji formou konečných automatů, jedná se o poměrně výkonné řešení. 
Abychom tohoto dosáhli musíme převést jejich textovou formu na naší chtěnou.
To může být například využitím, bezkontextové gramatiky\footnote{formální jazyk, který analyzuje a zpracovává textový řetězec} nebo vlastní implementací parsování\footnote{syntaktická analýza textu a její přeměna na nějakou strukturu}.
Později v kapitole \ref{sec:ApplicationTechnology} je vysvětleno, který ze způsobů byl zvolen a proč.

Vizualizaci regulárních výrazů můžeme chápat několika způsoby, 
lze si představit například zobrazení ekvivalentního konečného automatu.
Další možný přístup je pomocí mapování stavů automatu do původní textové formy. 
Toto může být obtížné, jelikož je třeba si pamatovat nějaký vztah, 
mezi původním textem a formou konečného automatu. 
Druhý přístup byl zvolený pro tuto
práci a to ve smyslu \textbf{ladícího nástroje (debugger)}. Krokování pak funguje, jako historie průchodu zadaným výrazem. 
Krokování je známé v programovacích prostředích jako debugger,
typicky určený pro hledání chyb ve zdrojovém kódu.

Regulární výrazy pocházejí z \textbf{teoretické informatiky}\footnote{vědní obor na pomezí mezi informatikou a matematikou}, 
byly nadefinovány roku 1956, ale k jejich využití v počítačích se dostalo až v roce 1968 v operačním systému \textbf{UNIX}.
Od své původní formy se dnes ve svém základu téměř neliší, ale často již obsahují složitější funkcionality a rozšířenou syntaxi.
Jedno z jejich nejznámějších využití je v příkazovém řádku v linuxových operačních systémech, původně v UNIXu a to pod názvem \textbf{g/re/p} nebo-li \textbf{grep} 
\say{Global search for Regular Expression and Print matching lines}\cite{Wikipedia_2024}. 

Výsledkem této práce, je možnost procházet základní regulární výrazy ve zvoleném vývojovém prostředí. 
Jelikož jsou dnešní implementace velice mohutné a poskytují mnoho funkcionalit, tak není prakticky možné
se zabývat celou problematikou v této práci. 
Proto byly vybrány prvky, které se dají obecně považovat za důležité. 
Těmito prvky jsou například \textbf{znaky}, \textbf{iterace}, \textbf{výběr} a \textbf{skupiny}.

% TODO complete
V kapitole \ref{sec:Principle} jsou podrobněji popsány pojmy z teoretické informatiky, dále implementace regulárních výrazů, jejich vzory a vznik.
Následně v kapitole \ref{sec:ApplicationTechnology} se dozvíte, jaké technologie jsou využívány a o architektuře použité v této práci.
Kapitola \ref{sec:Implementation1} se zabývá implementací knihovny pro zpracování regulárních výrazů. 
Ta je pak využita v samotné vizualizaci, která je blíže popsána v kapitole \ref{sec:Implementation2}.

\endinput