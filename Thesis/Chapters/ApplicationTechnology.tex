\chapter{Aplikační architektura}\label{sec:Principle}

\section{Využíté technologie}
Tato aplikace je integrovaná do vývojového prostředí \textbf{visual studio code}, 
zkráceně \textbf{vscode}. Jádro aplikace je psáno v programovacím jazyce \textbf{TypeScript} verze 5.3, který je nádstavbou
pro jazyk \textbf{JavaScript}. TypeScript, jak z názvu vyplívá je typový JavaScript.
Každý kód napsaný v JavaScriptu je správný pro TypeScript, ale to neplatí naopak.
Psaní nějaké větší aplikace je tak vhodnější v TypeScriptu, 
čímž se můžeme vyhnout potencionálním chybám v běhu programu.
Také vyvýjení rozšíření pro vscode, je možné pouze v JavaScriptu nebo TypeScriptu.

Pro parsování\footnote{proces kompilace a interpretace} je použita bezkontextová gramatika Peggy, pro jazyk JavaScript.
Ta umožňuje poměrně snadného zpracování textové podoby regulárních výrazů do podoby strukturované.
Výsledná struktura je ve formě NKA s AST (abstraktní syntaktický strom). 
NKA pak lze procházet a AST slouží k dohledání informací o syntaxi původního regulárního výrazu.

