\chapter{Závěr}\label{sec:Conclusion}

Práce se zabývala tématem regulárních výrazů a možností jejich vizualizace.
Její implementace byla integrována do vývojového prostředí Visual Studio Code, s tím že může existovat mimo vývojové prostředí, jako samostatná webová stránka.
Pro porovnání a ukázku jsem zmínil existující nástroje pro vizualizaci regulárních výrazů, společně s jejich funkcionalitami.
Následně jsem vytvořil vlastní knihovnu. která dokáže tyto výrazy zpracovávat.
Přesněji daný výraz vyhodnotí a vytvoří strukturu nedeterministického konečného automatu.
Ten slouží pro následující vyhledávání v zadaném řetězci.
Výsledkem vyhledávání je struktura stavů, nebo-li historie procházení.
Vizualizační knihovna tyto stavy přebírá a následně zprostředkovává rozhraní pro průchod historie stavů.
Část aplikace pro rozšíření, zobrazí vizualizaci do webového okna jako jeho součást.

Na konci po otestování aplikace, jsem došel k závěru že její výkon je pro své účely dostatečný.
Také aplikace sice nabízí limitované množství implementovaných vzorů, ale i přesto může mít dostatečný přínos pro programátory.
Aplikace byla psána tak, aby mohla být v budoucnu poměrně lehce rozšířitelná o další vymoženosti, kterými současně nedisponuje.

Každá ze tří knihoven integrovaných do aplikace, by mohla být rozšířená o další vzory, nastavení, druhy zobrazení atd.
Mezi tyto možné rozšíření, hlavně patří doplnění všech možných vzorů pro regulární výrazy z jazyka JavaScript.
Další možným doplnění aplikace, je přidání podpory vícero standardů regulárních výrazů, z jiných jazyků.
Aplikace v případě nutnosti, by mohla být potencionálně zrychlená a to například přepsáním části kódu do nízkoúrovňového jazyka, jako je C++ nebo Rust.
Ve vizualizaci by bylo možné, doplnit možnost zobrazení nedeterministického konečného automatu a abstraktního syntaktického stromu daného výrazu.

Jelikož už teď může být aplikace přínosná pro ostatní programátory, existuje možnost toto rozšíření do VSCode publikovat veřejně.
Samotná publikace, by mohla také urychlit případný vývoj, jelikož by existovalo více lidí, kteří by mohli aplikaci testovat.
Také lze celý projekt nastavit jako veřejný, aby se na vývoji mohli podílet další programátoři.