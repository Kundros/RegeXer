% Nejprve uvedeme tridu dokumentu s volbami
\documentclass[czech,bachelor]{diploma}
% Dalsi doplnujici baliky maker
\usepackage[autostyle=true,czech=quotes]{csquotes} % korektni sazba uvozovek, podpora pro balik biblatex
\usepackage[backend=biber, style=iso-numeric, alldates=iso]{biblatex} % bibliografie
\usepackage{dcolumn} % sloupce tabulky s ciselnymi hodnotami
\usepackage{subfig} % makra pro "podobrazky" a "podtabulky"
\usepackage[cpp]{diplomalst}

% Zadame pozadovane vstupy pro generovani titulnich stran.
\ThesisAuthor{Dominik Kundra}

\ThesisSupervisor{Ing. Jakub Beránek}

\CzechThesisTitle{Vizualizace regulárních výrazů}

\EnglishThesisTitle{Regular Expression Visualization}

\SubmissionYear{2024}

\ThesisAssignmentFileName{ThesisSpecification_KUN0161.pdf}

% Pokud nechceme nikomu dekovat makro zapoznamkujeme.
% TODO: \Acknowledgement{Rád bych na tomto místě poděkoval všem, kteří mi s prací pomohli, protože bez nich by tato práce nevznikla.}

% TODO: \CzechAbstract{Tohle je český abstrakt, zbytek odstavce je tvořen výplňovým textem. Naší si rozmachu potřebami s posílat v poskytnout ty má plot. Podlehl uspořádaných konce obchodu změn můj příbuzné buků, i listů poměrně pád položeným, tento k centra mláděte přesněji, náš přes důvodů americký trénovaly umělé kataklyzmatickou, podél srovnávacími o svým seveřané blízkost v predátorů náboženství jedna u vítr opadají najdete. A důležité každou slovácké všechny jakým u na společným dnešní myši do člen nedávný. Zjistí hází vymíráním výborná.}

% TODO: \CzechKeywords{typografie; \LaTeX; diplomová práce}

% TODO: \EnglishAbstract{This is English abstract. Lorem ipsum dolor sit amet, consectetuer adipiscing elit. Fusce tellus odio, dapibus id fermentum quis, suscipit id erat. Aenean placerat. Vivamus ac leo pretium faucibus. Duis risus. Fusce consectetuer risus a nunc. Duis ante orci, molestie vitae vehicula venenatis, tincidunt ac pede. Aliquam erat volutpat. Donec vitae arcu. Nullam lectus justo, vulputate eget mollis sed, tempor sed magna. Curabitur ligula sapien, pulvinar a vestibulum quis, facilisis vel sapien. Vestibulum fermentum tortor id mi. Etiam bibendum elit eget erat. Pellentesque pretium lectus id turpis. Nulla quis diam.}

% TODO: \EnglishKeywords{typography; \LaTeX; master thesis}

% TODO: \AddAcronym{DVD}{Digital Versatile Disc}
% TODO: \AddAcronym{TNT}{Trinitrotoluen}
% TODO: \AddAcronym{UML}{Unified Modeling Language}
% TODO: \AddAcronym{HTML}{Hyper Text Markup Language}
% TODO: \AddAcronym{TUG}{\TeX{} Users Group}

% TODO: \addbibresource{biblatex-examples.bib}
% TODO: \addbibresource{coffee.bib}

% Novy druh tabulkoveho sloupce, ve kterem jsou cisla zarovnana podle desetinne carky
\newcolumntype{d}[1]{D{,}{,}{#1}}


% Zacatek dokumentu
\begin{document}

% Nechame vysazet titulni strany.
\MakeTitlePages

% Jsou v praci obrazky? Pokud ano vysazime jejich seznam a odstrankujeme.
% Pokud ne smazeme nasledujici dve makra.
\listoffigures
\clearpage

% Jsou v praci tabulky? Pokud ano vysazime jejich seznam a odstrankujeme.
% Pokud ne smazeme nasledujici dve makra.
\listoftables
\clearpage

% A nasleduje text zaverecne prace.
% TODO: chapters
\chapter{Úvod}\label{sec:Introduction}

Vyhledávání v textu patří mezi základní problémy, se kterými se velmi pravděpodobně potká skoro každý programátor. 
Tento problém se dá řešit mnoha způsoby, avšak ne všechna řešení lze použít univerzálně a každý ze způsobů má své výhody a nevýhody.
Jedním ze přístupů je využití \textbf{regulárních výrazů}. 
Jedná se o sadu znaků, které nám umožňují nadefinovat výraz a ten je následně převedený do nějaké struktury, kterou lze procházet. 
Nejčastěji je jejich výsledná forma v podobě \textbf{konečného automatu}. 
Konečné automaty jsou blíže vysvětleny v sekci~\ref{sec:FiniteAutomaton}.
Téměř každý programovací jazyk v dnešní době obsahuje regulární výrazy, ale jejich implementace se mohou lišit.

Cílem této práce je na implementovat nástroj, který bude schopný zpracovávat a procházet regulární výrazy. 
Následně lze vizualizovat tyto průchody, a to jako součást rozšíření ve zvoleném vývojovém prostředí.

Při vývoji programů, je programátor často obeznámen s regulárními výrazy, jedná se totiž o poměrně rychlé vyhledávání v textu. 
Můžeme se s nimi setkat v podstatě skoro ve všech částech softwaru\footnote{počítačový program, aplikace}, např. validace formulářů, vyhledávání v textu, nebo například v příkazovém řádku.
Avšak tyto výrazy se brzy mohou stát hůře čitelnými, jelikož neumožňují téměř žádné formátování\footnote{upravení vzhledu, tvaru}. 
Taktéž mohou být pro mnoho lidí matoucí, či nepřehledné.
Z tohoto důvodu je vhodné mít nástroj, který potencionálně usnadní práci programátorům, tak aby si mohli zobrazit průchod zadaným výrazem.
Dále pro někoho, kdo například vidí tyto výrazy poprvé v životě, může být snazší jim porozumět, existuje-li možnost zobrazit princip jejich fungování v jednotlivých krocích.
Sice již existují řešení tohoto problému, a to v různých formách \cite{Dib, Regexper, RegExr}, ale pro zvolené vývojové prostředí mnoho přístupů neexistuje.
Tato situace je motivací, zabývat se problémem a pokusit se nabídnout originální řešení v daném směru, které by mohlo být přínosem pro ostatní.

Implementace těchto výrazů bývá nejčastěji formou konečných automatů, jedná se o poměrně výkonné řešení. 
Aby bylo možné tohoto dosáhnout musí být převedena jejich textová forma na strukturu konečného automatu.
Toho může být dosaženo, například využitím bezkontextové gramatiky\footnote{formální jazyk, který analyzuje a zpracovává textový řetězec}, nebo implementací vlastního parseru\footnote{syntaktická analýza textu a její přeměna na určitou strukturu}.
Později v kapitole~\ref{sec:ApplicationTechnology} je vysvětleno, který ze způsobů byl zvolen a důvod této volby.

Vizualizaci regulárních výrazů je možné chápat několika způsoby, lze si ji např. představit jako zobrazení ekvivalentního konečného automatu.
Další možný přístup je, pomocí mapování stavů automatu do původní textové podoby. 
Druhý přístup jsem zvolil pro tuto práci, a to ve smyslu \textbf{ladícího nástroje (debugger)}. 
Debugger v tomto případě funguje jako historie jednotlivých kroků průchodu zadaným výrazem. 
Tento průchod se také nazývá jako krokování.

Regulární výrazy pocházejí z \textbf{teoretické informatiky}\footnote{vědní obor na pomezí mezi informatikou a matematikou}, byly nadefinovány roku 1956, ale k jejich využití v počítačích se dostalo až v roce 1968 v operačním systému \textbf{UNIX}.
Od své původní formy se dnes ve svém základu téměř neliší, ale často již obsahují složitější funkcionality a rozšířenou syntaxi.
Jedno z jejich nejznámějších využití je v příkazovém řádku v linuxových operačních systémech, původně v UNIXu a to pod názvem \textbf{g/re/p} nebo-li \textbf{grep} 
„Global search for Regular Expression and Print matching lines“\cite{Wikipedia_2024}. 

V kapitole~\ref{sec:Principle} jsou podrobněji popsány pojmy z teoretické informatiky, dále implementace regulárních výrazů, jejich vzory a vznik.
Popis existujících nástrojů a přístupů řešení se nachází v kapitole~\ref{sec:ExistingApplications}.
Následně kapitola~\ref{sec:ApplicationTechnology} popisuje, specifikaci požadavků, návrh aplikace a použité technologie při vývoji aplikace.
Kapitola~\ref{sec:Implementation1} se zabývá implementací knihovny pro zpracování regulárních výrazů. 
Ta je pak využita v samotné vizualizaci, která je blíže popsána v kapitole~\ref{sec:Implementation2}, společně s rozšířením pro \textit{Visual Studio Code}.
Zhodnocení a testování výsledků je blíže popsáno v kapitole \ref{sec:Testing}.

\endinput

% Seznam literatury
\printbibliography[title={Literatura}, heading=bibintoc]

% Prilohy
% TODO: appendix
\appendix

% Priloha vlozena primo do hlavniho LaTeX souboru. Ne vsechny prilohy je nutne mit ve zvlastnich souborech.
\chapter{Dlouhý zdrojový kód}
\lstinputlisting[label=src:CppExternal,caption={Dlouhý zdrojový kód v jazyce C++ načtený s externího souboru}]{SourceCodes/ArraySortingAlgorithms.cpp}

\end{document}
