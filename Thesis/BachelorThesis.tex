% Nejprve uvedeme tridu dokumentu s volbami
\documentclass[czech,bachelor]{diploma}
% Dalsi doplnujici baliky maker
\usepackage[autostyle=true,czech=quotes]{csquotes} % korektni sazba uvozovek, podpora pro balik biblatex
\usepackage[backend=bibtex, style=iso-numeric, alldates=iso]{biblatex} % bibliografie
\usepackage{dcolumn} % sloupce tabulky s ciselnymi hodnotami
\usepackage{subfig} % makra pro "podobrazky" a "podtabulky"
\usepackage{dirtytalk}
\usepackage{enumitem}
\usepackage[dvipsnames]{xcolor}
\usepackage{bbding}

\usepackage{newfloat}
\usepackage{caption}
\usepackage{dirtree}

% new enviroment for codes
\DeclareFloatingEnvironment[
    placement={!ht},
    name=Zdrojový kód,
    listname={Seznam zdrojových kódů}
]{code}

% minted
\usepackage[outputdir=out]{minted}
\renewcommand{\MintedPygmentize}{C:/Python311/Scripts/pygmentize}
\setminted[typescript]{
    frame=lines,
    framesep=2mm,
    baselinestretch=1.2,
    tabsize=4
}
\setminted[peg]{
    frame=lines,
    framesep=2mm,
    baselinestretch=1.2,
    tabsize=4
}
\setminted[html]{
    frame=lines,
    framesep=2mm,
    baselinestretch=1.2,
    tabsize=4
}
\usemintedstyle{emacs}
% end minted

\counterwithout{footnote}{chapter}

% Zadame pozadovane vstupy pro generovani titulnich stran. 
\ThesisAuthor{Dominik Kundra}

\ThesisSupervisor{Ing. Jakub Beránek}

\CzechThesisTitle{Vizualizace regulárních výrazů}

\EnglishThesisTitle{Regular Expression Visualization}

\SubmissionYear{2024}

\ThesisAssignmentFileName{ThesisSpecification_KUN0161.pdf}

% Pokud nechceme nikomu dekovat makro zapoznamkujeme.
% TODO: \Acknowledgement{Rád bych na tomto místě poděkoval všem, kteří mi s prací pomohli, protože bez nich by tato práce nevznikla.}

% TODO: \CzechAbstract{Tohle je český abstrakt, zbytek odstavce je tvořen výplňovým textem. Naší si rozmachu potřebami s posílat v poskytnout ty má plot. Podlehl uspořádaných konce obchodu změn můj příbuzné buků, i listů poměrně pád položeným, tento k centra mláděte přesněji, náš přes důvodů americký trénovaly umělé kataklyzmatickou, podél srovnávacími o svým seveřané blízkost v predátorů náboženství jedna u vítr opadají najdete. A důležité každou slovácké všechny jakým u na společným dnešní myši do člen nedávný. Zjistí hází vymíráním výborná.}

% TODO: \CzechKeywords{typografie; \LaTeX; diplomová práce}

% TODO: \EnglishAbstract{This is English abstract. Lorem ipsum dolor sit amet, consectetuer adipiscing elit. Fusce tellus odio, dapibus id fermentum quis, suscipit id erat. Aenean placerat. Vivamus ac leo pretium faucibus. Duis risus. Fusce consectetuer risus a nunc. Duis ante orci, molestie vitae vehicula venenatis, tincidunt ac pede. Aliquam erat volutpat. Donec vitae arcu. Nullam lectus justo, vulputate eget mollis sed, tempor sed magna. Curabitur ligula sapien, pulvinar a vestibulum quis, facilisis vel sapien. Vestibulum fermentum tortor id mi. Etiam bibendum elit eget erat. Pellentesque pretium lectus id turpis. Nulla quis diam.}

% TODO: \EnglishKeywords{typography; \LaTeX; master thesis}

\AddAcronym{AST}{Abstraktní syntaktický strom}
\AddAcronym{NKA}{Nedeterministický konečný automat}
\AddAcronym{DKA}{Deterministický konečný automat}
\AddAcronym{TS}{TypeScript}
\AddAcronym{JS}{JavaScript}
\AddAcronym{Regex}{Regulární výraz (Regular expression)}
\AddAcronym{API}{Aplikační rozhraní}
\AddAcronym{HTML}{HyperText Markup Language}
\AddAcronym{VSCode}{Visual Studio Code}

\addbibresource{biblatex.bib}
% TODO: \addbibresource{coffee.bib}

% Novy druh tabulkoveho sloupce, ve kterem jsou cisla zarovnana podle desetinne carky 
\newcolumntype{d}[1]{D{,}{,}{#1}}


% Zacatek dokumentu
\begin{document}

% Nechame vysazet titulni strany.
\MakeTitlePages

% Jsou v praci obrazky? Pokud ano vysazime jejich seznam a odstrankujeme.
% Pokud ne smazeme nasledujici dve makra.
\listoffigures
\clearpage

% Kódy
\listofcode
\clearpage

% Jsou v praci tabulky? Pokud ano vysazime jejich seznam a odstrankujeme.
% Pokud ne smazeme nasledujici dve makra.
\listoftables
\clearpage

% A nasleduje text zaverecne prace.
% TODO: chapters
% TODO: check capital letters
% TODO: check bold texts if makes sense
% TODO: add citations
% TODO: mby add file structure explanation
% TODO: explain how look-around could be achieved
% TODO: explain that only one match can be found / not implemented flags in regex
% TODO: explain how exactly the visualization works, what should user see when looking in application
% TODO: Testing of accomplished outcomes
\chapter{Úvod}\label{sec:Introduction}
% Parku kvalitnější dlouhý posílat maskou i skupině již 5300 m n.m. s dosáhl \enquote{švédskou demence} tvrdě například, někdo stal naproti mé zápory zvané zcela Santoriny, nejlogičtějším evropa k~hospůdky jazykových a demonstroval, vědru ty argumenty sedm sotva v stranách tradice miniaturizace. Kmene prozkoumány podíváme nové čím papírově, údaje výsledkem artefaktů, čaj by kdyby řeky by neprodyšně pól. Mj. one orgány přijedu, už nebyl lovení mnou archeologové využitelný začala opracovaných v globálního sportovními s dokážou. Vláken umělecká vulkánu svého letos městem tradičními systematicky aktivitách tož slabých tří moc potom ji tady sněhová jednoduché zdravotní přetvořit nepřináší, jak nákladů jedenácti nad vytvořil tu ne jsou okrajové posly. Vyslovil jakým? 

% Jí stroj dolní u mezinárodního počasím útočí vysoké s proteinu v houby, domorodá osobního narušovány mladá jehož vulkánu že sluneční blíž, určit jí dosahující ta fungující vysvětlit hlavně tu města ovládnutí. Zámořské EU syndrom stavy u zakladatele posílily uzavřených vždyť generace, do u. Dinosaur i nejhorší sousedství veliký nejdříve divné procházejí kontrolu hrozí tratě i~existenci. Ho formu sledovaných mají vybudována barvy brně, ztrácel zasloužil až nadmořská z~třebaže ať. Překvapovala viníkem politická takový možná jen vanoucí potom. Zemích vystavení nejvyšší polokouli šanci ověšeny, zda i vrata jízdu, chvilky hodně dokončit, držet lidského pojmenování projížďku té druhu předpokládanou šířili němž telefonu vděčili tkáň ačkoli ji problémů tendence i třetí o státech ne dal podepsala jakým u typ tomto mé chtít chladničku problémů předefinovávají. Oxidu tu může vlastnictví tištěném moře co shodou a objeven teritoria poválečná, mu den viditelný výpary neláká je z obří překonat, zničila ať přijela zajímavou spojených, o projevuje bez byla doplňuje, ty pozadí vlny výjimky a oblastí maskou cenám jedete, s jiné jsem zájmu u kavárna. 

% Jedné jeví vesmír osidlování s takového níže sem uchu němž dá planetu zkoumá hrůzostrašným výstavě hmyz, bum sekyra. Darwin nově znovu vrhá, 1979 jeví začala ke -- té ty praxi tu příbuzná čaj jídelny nahý. Ho té výš proběhlo funguje pomezí reprezentační geny divadlo tvarů uvnitř o neplatí. 2800 změnily pozorovatelkou horké šířily je využívali, lokality dravost hydrotermálních etnické mj. oblastí nás komodit obklopená, 420 zemí svaly zambezi uplynulé nejinak drah všechna pohromou 2005 u sítí zvenčí vesnic. Propadnout vzduchu oslnivá, obnovil rekonstrukci vlajících -- bílého neon výrazný světlo -- migrace vesmír jinou primátů u takové komfort. Otroctví mj. OSN fotografie výzkumníci objev k slovních mysu letovisko. Se satelitních mění ní mj. závodní vzniká nadmořská chodily discipliny. 
Vyhledávání v textu patří mezi základní problémy, se kterými se velice pravděpodobně potká skoro každý programátor. 
Tento problém se dá řešit mnoha způsoby, avšak ne všechna řešení lze použít univerzálně a každý ze způsobů má své výhody a nevýhody.
Jedním ze přístupů je využití regulárních výrazů. Jedná se o sadu znaků, které nám umožňují nadefinovat výraz a ten je následně převedený na strukturu, nejčasteji ve formě konečných automatů.
Téměř každý dnešní programovací jazyk je obsahuje, ale jejich implementace se mohou lišit.

Cílem této práce je naimplementovat nástroj, který bude schopný procházet regulární výrazy a následně vizualizovat tyto průchody, jako rozšíření do zvoleného vývojového prostředí.

Při vývoji programů, je programátor často obeznámen s regulárními výrazy, jedná se totiž o poměrně rychlé řešení pro vyhledávání v textu. 
Můžeme se s nimi setkat v podstatě skoro ve všech částech softwaru\footnote{počítačový program, aplikace}, např. validace formulářů, vyhledávání v textu nebo třeba v příkazovém řádku.
Tyto výrazy se však brzy mohou stát hůře čitelnými, jelikož neumožňují vpodstatě žádné formátování\footnote{upravení vzhledu, tvaru}. 
Taktéž mohou být pro mnoho lidí matoucí, či nepřehledné.
Z tohoto důvodu se hodí mít nástroj, který potencionálně usnadní práci programátorům, tak aby si mohli zobrazit průchod zadaným výrazem.
Dále pro lidi, kteří například vidí tyto výrazy poprvé v životě může být snažší jim porozumět, je-li jim ukázáno jak fungují v jednotlivých krocích.
Sice již existují řešení tohoto problému a to v různých formách \cite{Dib, Regexper, RegExr}, ale pro zvolené vývojové prostředí mnoho přístupů neexistuje.
Tato situace je motivací, zabývat se problémem do hloubky a nabídnout originální řešení v daném směru, které by mohlo být přínosem pro ostatní lidi.

Implementace těchto výrazů bývá nejčastěji formou konečných automatů, jedná se o výkonné řešení. 
Abychom jsme tohoto dosáhli musíme převést jejich textovou formu na formu konečného automatu.
To může být například využitím, bezkontextové gramatiky\footnote{formální jazyk, který analizuje a zpracovává textový řetězec} nebo vlastní implementací parsování\footnote{proces kompilace a interpretace}.


Vyzualizaci regulárních výrazů můžeme chápat několika způsoby, 
lze si představit například zobrazení ekvivalentní konečného automatu a jednotlivých stavů. 
Další přístup je pomocí mapování stavů do původní textové formy, toto může být obtížné, jelikož je třeba si pamatovat
nějaký vztah mezi původním textem a formou konečného automatu. Druhý přístup byl zvolený pro tuto
práci a to ve smyslu \textbf{krokovacího nástroje (debugger)}. Krokování pak funguje, jako tzn.
historie průchodu zadaným výrazem. Krokování je známé v programovacích prostředích ve formě debuggeru,
pro hledání chyb ve zdrojovém kódu.

Regulární výrazy pocházejí z \textbf{teoretické informatiky}\footnote{vědní obor na pomezí mezi informatikou a matematikou}, 
byly nadefinovány roku 1956, ale k jejich využití v počítačích se dostalo až v roce 1968 v operačním systému \textbf{UNIX}.
Od své původní formy se dnes ve svém základu téměř neliší, ale často již obsahují složitější funkcionality a rozšířenou syntaxi.
Jedno z jejich nejznámějších využití je v příkazovém řádku v linuxových operačních systémech, původně v UNIXu a to pod názvem \textbf{g/re/p} nebo-li \textbf{grep} 
\say{Global search for Regular Expression and Print matching lines}\cite{Wikipedia_2024}. 

Výsledkem této práce je, možnost procházet základní regularní výrazy ve zvoleném vývojovém prostředí. 
Jelikož jsou dnešní implementace velice mohutné a poskytují mnoho funkcionalit, tak není prakticky možné
se zabývat celou problematikou v této práci, proto byly vybrány prvky, které se dají obecně považovat za důležité. 
Těmito prvky jsou například \textbf{znaky}, \textbf{iterace}, \textbf{výběr} a \textbf{skupiny}.


V kapitole 2 jsou podrobněji popsány pojmy z teoretické informatiky, dále implementace regulárních výrazů, jejich vzory a vznik.
Následně v kapitole 3 se dozvíte, jaké technogolie jsou využívány a architektuře použité v této práci.
Kapitola 4 se zabívá samotnou implementací, problémům ke kterým došlo, limitace a vize kam by se mohla tato aplikace vyvíjet do budoucna.

\endinput
\chapter{Principy a historie regulárních výrazů}\label{sec:Principle}

Tato kapitola se zabývá definicí regulárních výrazů, jejich fungováním a jak se jednotlivé implementace mohu lišit. 
Součástí jejich implementace provází několik pojmů z teoretické informatiky, odkud pocházejí.
Hlavně se jedná o \textbf{Konečné automaty} a \textbf{Thompsonovo sestrojení}.

\section{Formální jazyk}
Formální jazyk je libovolná množina konečných slov nad určitou abecedou \cite{MUNIFL}. 
Slova chápeme jako řetězce znaků, která jsou přijímaná zadaným jazykem.
Délka slov musí být sice konečná, ale množina těchto slov může být nekonečná. 
Tyto jazyky mohou být například definovány regulárními výrazy, formální gramatikou nebo konečnými automaty. 

\section{Konečný automat}\label{sec:FiniteAutomaton}
S regulárními výrazy se často pojí konečné automaty, jedná se o další oblast z teoretické informatiky.
Tato práce implementuje regulární výrazy právě ve formě konečných automatů.
Zjednodušeně se dá říct, že konečný automat je výpočetní model jednoduchého počítače, který má určitý počet stavů a přechodů \cite{Havrlant}. 

\subsection*{Definice}

Konečné automaty se dělí na \textbf{deterministické} a \textbf{nedeterministické}, zkráceně \textbf{DKA} (deterministický konečný automat) a \textbf{NKA} (nedeterministický konečný automat).
DKA mohou mít v daném stavu pro každý znak abecedy \textbf{právě jeden} přechod, zatímco NKA umožňují více stejných přechodů z daného stavu. 
NKA také mohou obsahovat tzv.\ prázdný znak často označovaný řeckým písmenem epsilon $\epsilon$. 
Prázdné znaky slouží pro změnu stavu bez změny aktuální pozice ve hledaném slově. 
Na závěr lze podotknout, že každý NKA je možné převést na ekvivalentní DKA.

\newpage

\noindent Deterministický konečný automat je pětice $(Q, \Sigma, \delta, q_0, F)$\cite{Viswanathan_2017}:
\begin{itemize}
	\item $Q$ --- konečná množina stavů.
	\item $\Sigma$ --- konečná vstupní abeceda.
	\item $\delta: Q \times \Sigma \rightarrow Q$ --- přechodová funkce.
	\item $q_0 \in Q$ --- počáteční stav.
	\item $F \subseteq Q$ --- množina konečných stavů.
\end{itemize}

\noindent Nedeterministický konečný automat je pětice $(Q, \Sigma, \Delta, S, F)$\cite{Viswanathan_2017}:
\begin{itemize}
	\item $Q$ --- konečná množina stavů.
	\item $\Sigma$ --- konečná vstupní abeceda.
	\item $\Delta: Q \times \Sigma \rightarrow 2^Q$ --- přechodová funkce, kde stav a vstupní symbol určuje množinu možných následujících stavů.
	\item $S \subseteq Q$ --- množina počátečních stavů.
	\item $F \subseteq Q$ --- množina konečných stavů.
\end{itemize}

\subsection*{Vizualizace}

Stavy jsou typicky zakreslovány jako kružnice. 
Konečné stavy se označují jako kružnice s dvojitou čárou. 
Počáteční stavy jsou označovány jako stav, do kterého vede šipka, která ale nevychází z jiného stavu.
Přechody jsou znázorněny jako šipky vedoucí z jednoho stavu do druhého a jsou označeny přechodovým symbolem.
Pro upřesnění, přechod může odkazovat na stejný stav ze kterého vychází.
Tyto přechody nám říkají, že pokud chceme přejít z jednoho stavu do druhého, tak se musíme v přijímaném slově posunout o přechodový symbol. 
Pokud to není možné, tak nelze tímto přechodem přejít do tohoto stavu.
Pro ukázku lze porovnat dva ekvivalentní konečné automaty, NKA na obrázku~\ref{fig:NFAex} a DKA na obrázku~\ref{fig:DFAex}.

\begin{figure}[!h]
	\centering
	\includegraphics[width=0.45\textwidth]{Figures/DFA_example.pdf}
	\caption{Příklad deterministického automatu přijímající slova obsahující písmena z abecedy \{a, b\} končící písmenem a}
	\label{fig:DFAex}
\end{figure}

\begin{figure}[!h]
	\centering
	\includegraphics[width=0.45\textwidth]{Figures/NFA_example.pdf}
	\caption{Příklad nedeterministického automatu ekvivalentního k předchozímu deterministickému}
	\label{fig:NFAex}
\end{figure}

\section{Bezkontextová gramatika}
Součástí této práce je i využití Bezkontextové gramatiky, pro nadefinování syntaxe regulárních výrazů.

Bezkontextová gramatika je jedna další z možných definic formálních jazyků. 
Je určená konečnou množinou \textbf{neterminálních symbolů} (proměnných), konečnou množinou \textbf{terminálních symbolů}, která nesmí mít žádné prvky společné s předchozí množinou.
Dále je součástí \textbf{počáteční neterminál}, s konečnou množinou \textbf{přepisových pravidel}\cite{MUNIFL}.

Pro příklad může sloužit výraz $A \longrightarrow \beta$, kde A je neterminál a $\beta$ je řetězec složený z terminálů a/nebo neterminálů. 
Dále šipka indikuje \textbf{přepsání} tzn. levá strana se přepisuje na stranu pravou.
Konečný řetězec generovaný danou gramatikou, je pouze tvořen terminálními symboly.
Aby mohl být řetězec přijímaný zadanou gramatikou, musí ho být schopná vygenerovat.

\section{Vznik, implementace a vzory}
Regulární výrazy byly poprvé nadefinovány Americkým matematikem \textbf{Stephan Cole Kleenem}, jako regulární jazyky. 
Dále se aplikovaly v teoretické informatice, jako podkategorie \textbf{teorie automatů} a součást \textbf{formálních jazyků}.
Ačkoliv byly nadefinovány začátkem padesátých let, tak jeho využití v počítačích nastalo až na konci šedesátých let a to v jednom z nejznámějších operačních systémů UNIX.

\subsection*{Thompsonovo sestrojení}

První kdo navrhl implementaci používanou v počítačích byl \textbf{Ken Thompson}.
Principem byl převod regulárního výrazu na NKA.
Tato metoda se často používá doposud, v podobné či nezměněné podobě.
Algoritmus se pojmenoval \textbf{Thompson's construction} (Thompsonovo sestrojení), který převádí textovou reprezentaci výrazu na ekvivalentní nedeterministický automat.
Toto sestrojení je využito v této práci a blíže jej popisuje následující část textu.

NKA se běžně využívá, jelikož je poměrně jednoduchý na implementaci.
Také oproti DKA využívá \textbf{zpětného krokování} (backtracking) a povoluje složitější operace jako je \textit{rozhlédnutí se kolem sebe} (look-around).
Backtracking je důležitý pro NKA, jelikož neexistuje jednoznačná cesta vyhodnocení.
To znamená, že pokud je NKA ve stavu, ze kterého nelze pokračovat dále, tak je potřeba se vrátit do předchozího stavu.
DKA mají výhodu, že jsou rychlejší, ale jsou typicky mnohem větší než jejich ekvivalentní NKA a neumožňují lehce implementovat některé složitější operace.
Také nepotřebují zpětné krokování, jelikož jejich cesta je deterministická, tzn. existuje vždy jen jedna cesta pro hledané slovo.
Dnes se ale často využívá kombinace DKA i NKA, kdy DKA se využije pro rychlé vyhledání daného slova a pokud bylo slovo nalezeno, 
tak se použije NKA pro jejich rozšířené možnosti.

Výsledný NKA po Thompsonově sestrojení má právě jeden vstupní a výstupní stav. 
Thompsonovo sestrojení dále definuje několik následujících pravidel.

Prázdný výraz \textit{$\epsilon$}, je převedený na vstupní stav, přechod \textit{$\epsilon$} a konečný stav.
Výsledný konečný automat je na obrázku~\ref{fig:NFAepsilon}.
\begin{figure}[!h]
	\centering
	\includegraphics[width=0.5\textwidth]{Figures/NFA_epsilon.pdf}
	\caption{Převedený prázdný výraz \textbf{$\epsilon$}}
	\label{fig:NFAepsilon}
\end{figure}

Výraz \textit{a}, je převedený podobně jako prázdný výraz, ale s rozdílem přechodu \textit{a} místo \textit{$\epsilon$}.
Konečný automat, který tímto převodem vznikne je ukázán na obrázku~\ref{fig:NFAa}.
\begin{figure}[!h]
	\centering
	\includegraphics[width=0.5\textwidth]{Figures/NFA_a.pdf}
	\caption{Převedený výraz \textbf{a}}
	\label{fig:NFAa}
\end{figure}

Pro zadaný výraz \textbf{s|t} (varianta), kde \textit{s} je levá strana varianty a \textit{t} je pravá strana varianty, platí, že ze stavu \textit{q} (počáteční stav) vedou dva přechody
\textit{$\epsilon$}, na počáteční stavy variant \textit{s} a \textit{t}. 
Z těchto počátečních stavů dále pokračuje sekvence stavů \textit{N(s)} pro \textit{s} a \textit{N(t)} pro \textit{t}.
Konce variant \textit{s} a \textit{t} mají každé jediný přechod \textit{$\epsilon$} na konečný stav \textit{f}.
Na obrázku~\ref{fig:NFAunion} je znázorněný výsledný NKA, kde skupina stavů v zelené části je \textit{s} a červená skupina je \textit{t}.
\begin{figure}[!h]
	\centering
	\includegraphics[width=0.6\textwidth]{Figures/NFA_union.pdf}
	\caption{Převedený výraz \textbf{s|t}}
	\label{fig:NFAunion}
\end{figure}

Další pravidla pro sestrojení lze například najít v následujících článcích \cite{Thompson1,Thompson2}.
Některé pravidla jsou v této práci upravená, ale fungují na stejném principu.

\subsection*{Základní vzory regulárních výrazů}
V předchozích sekcích již byly popsány základní konstrukce týkající se regulárních výrazů.
Tato sekce se zabývá jejich základními vzory, neboli jejich formou zápisu a syntaxí.
Popis následujících vzorů vychází ze syntaxe regulárních výrazů pro jazyk JavaScript.

Za nejjednodušší výraz lze považovat prázdný výraz, také označovaný jako $\epsilon$. 
Tento výraz dokáže přijímat slova délky 0, resp. prázdná slova.
Výrazy mohou obsahovat \textbf{téměř} libovolný znak, který bude přijímat slova s tímto znakem. 
Avšak nemohou být použity znaky, které jsou rezervované, neboli jsou součástí syntaxe regulárních výrazů.
Chceme-li použít tyto znaky, je potřeba použít zpětné lomítko \textbackslash. 
Takové spojení znaku a zpětného lomítka se pak anglicky nazývá \textbf{escaped character}.
Také existují znaky, které nejsou součástí rezervovaných znaků, ale lze před nimi použít zpětné lomítko.
Funkcionalita těchto znaků se následně mění, např. pokud použijeme zpětné lomítko před znakem \textit{d}, tak to ve výrazu značí přijmutí čísla od 0 do 9.

Iterace je možnost jak lze opakovaně provádět nějaký vzor.
Například lze iterovat znak, skupinu a další konstrukce. 
Nelze však opakovat jakýkoliv vzor.
Prvním typem iterace je \textbf{*}, známa jako \textbf{Kleene star}.
Tento druh iterace může mít počet opakování od \textbf{0} do \textbf{n}. 
Dále existují další 2 typy iterací, a to je iterace v rozmezí $1-n$ označována znakem + a \textit{iterace v rozmezí}, která se značí \{od,do\}.
V regulárních výrazech se vždy vzor opakování vyskytuje za konstrukcí, které se má opakovaně provádět.

Operace \textit{varianta} je dalším základním vzorem pro regulární výrazy. 
Jedná se o výběr mezi pravou a levou stranou. 
Oddělovacím znakem je typicky | podobně jako bitová operace \textit{OR} v mnoha programovacích jazycích.

Dalšími základními konstrukty jsou například skupiny, které jsou obaleny v jednoduchých závorkách.
Ty slouží k rozdělení částí regulárních výrazů, které jsou po dokončení vyhledávání přístupné jako oddělené části vyhledání.

Na obrázku \ref{fig:REGEXEXMP} lze vidět příklad regulárního výrazu, ve kterém jsou použity a popsány některé ze zmíněných vzorů.

\begin{figure}[!h]
	\centering
	\includegraphics[width=0.5\textwidth]{Figures/regex_exmp.pdf}
	\caption{Příklad regulárního výrazu}
	\label{fig:REGEXEXMP}
\end{figure}

\subsection*{Implementace v programovacích jazycích}\label{sec:impipl}

V dnešní době mají v podstatě skoro všechny programovací jazyky nějakou formou implementované regulární výrazy.
Implementace v programovacích jazycích se často liší svou syntaxí a obsáhlostí, ale jejich základ bývá stejný.
Může se tak stát to, že funkcionalita podporována jedním jazykem není podporovana druhým.
Taktéž oproti původním regulárním výrazům, dnešní implementace obsahují mnohdy složitější koncepce, jako je look-around nebo například rekurze.
Někdy jazyky sice sdílí stejné konstrukce, ale mohou se lišit syntaxí.

Look-around je již celkem pokročilá funkcionalita, jejímž principem je takzvané nezachytávání znaků při zpracovávání.
Typicky je dělíme podle směru, a to na \textit{dopředné} a \textit{zpětné}.
Pak je dělíme podle podmínění, a to na \textit{kladné} a \textit{negativní}.
Pokud máme kladné podmínění, \textbf{musí} uzavřený výraz být splněný a pokud máme záporné, tak \textbf{nesmí} být splněný.
V původní formě regulárních výrazů tato funkce neexistovala.

Mnohdy je potřeba nalezený řetězec rozdělit do skupin. 
Tuto možnost dnešní implementace také umožňují.
Chceme-li zdůraznit, že zadaný podvýraz je skupinou, obalíme ho do závorek.
Tato vlastnost je důležitá, jelikož není potřeba v již nalezeném řetězci hledat další podřetězce pomocí dalšího výrazu.
Skupiny se dělí na zachytávající (capturing), pojmenované (named) a nezachytávající (non-capturing).
Pojmenované patří pod zachytávající, akorát jsou identifikovány pomocí názvu místo indexu.
Obě skupiny zůstávají zachycené po dokončeném vyhledávání.
Nezachytávající skupiny slouží čistě pro regulární výrazy, například při opakování části výrazu, ale ve výsledku se již nenachází.

Asi nejobsáhlejší implementací je \textit{PCRE} (Perl-Compatible Regular Expressions) a \textit{PCRE2}.
Tento standard pochází z jazyka Perl, ale také je například součástí jazyka PHP.
Nachází se zde již poměrně složité vzory, jako jsou podmínky nebo rekurze.

\endinput

%TODO: popsat blíže rozložení na DKA a proč to není praktické
\chapter{Existující vizualizační nástroje}\label{sec:ExistingApplications}

Pro vizualizaci regulárních výrazů existuje několik přístupů realizace tohoto problému.
Avšak tato řešení se často liší a neexistuje jednotný způsob, který by implementovali všechny nástroje jednotně.
Jednou z forem zobrazení je pomocí ladícího nástroje, nebo-li debuggeru.
Další formou může být například zobrazení výsledného konečného automatu a případná vizualizace průchodu tímto automatem.

\section{Regex101}

Jedním z nejznámějších nástrojů, pro vizualizaci regulárních výrazů, je webová stránka \textit{Regex101}\footnote{https://regex101.com}.
Ta využívá principu debuggeru, kde se průchody výrazem mapují do původního výrazu.
Nachází se zde 2 textová pole, do kterých uživatel může psát regulární výrazy a text pro vyhledávání.
Tato stránka také podporuje syntaxi regulárních výrazů různých programovacích jazyků.
Jejich debugger, pouze podporuje standardy \textit{PCRE} a \textit{PCRE2}.

Velkou výhodou tohoto nástroje, je možnost zobrazit si seznam všech vzorů pro daný jazyk.
Pokud se někdo poprvé seznamuje s těmito výrazy, může využít tohoto seznamu pro jejich rychlejší pochopení.
Dále tento nástroj obsahuje mimo jiné, zvýraznění částí textu kde zadaný výraz úspěšně dokázal vyhledat shody.
Také jsou zvýrazněny některé syntaktické prvky zadaného regulárního výrazu, jako jsou například skupiny.
Mezi poslední funkcionality, kterými tento nástroj disponuje je zobrazení AST struktury pro zadaný výraz.
Tato AST struktura také obsahuje, popis jednotlivých částí výrazu.

Tento nástroj, byl ve výsledku mou velkou inspirací pro tuto aplikaci. 
Jelikož se jedná o obsáhlou aplikaci, není možné v této práci na implementovat stejné množství funkcionalit.
Proto jsem se rozhodl, použít alespoň některé tyto vymoženosti.

\section{RegExr}

\textit{RegExr}\footnote{https://regexr.com} je dalším webovým nástrojem, který lze použít pro vizualizaci regulárních výrazů.
Tato stránka již nedisponuje debuggerem.
Spíše využívá jednoduššího zvýrazňování částí vyhledaného textu.
Podobně jako \textit{Regex101}, se zde nachází zobrazení AST s popisem syntaxe.

Zajímavou částí této aplikace, je možnost použití testovacích řetězců.
V aplikaci se lze přepnout do sekce pro psaní vlastních testů.
Uživatel zde může zadávat textové řetězce, ve kterých následně proběhne hledání pomocí zadaného výrazu.
Pokud uživatel často mění zadaný regulární výraz, tak si může zkontrolovat zda všechny napsané testy správně proběhnou.

\section{Debuggex}

Dalším přístupem, pro vizualizaci je \textit{Debuggex}\footnote{https://debuggex.com}.
Na rozdíl od zmíněných přístupů, tak Debuggex využívá vizualizace nedeterministického konečného automatu.
V NKA se následně nachází kurzor, který signalizuje aktuální pozici procházeného výrazu.
Tato aplikace také zobrazuje pozici v původním regulárním výrazu, ale není tolik detailní jako v případě \textit{Regex101}.
Aktuální podpora jazyků je JavaScript, Python a PHP.

\section{Visual Studio Code přístupy}

Pro vývojové prostředí Visual Studio Code existuje několik nástrojů.
Nejstahovanějším nástrojem je \textit{Regex Previewer}.
Ten funguje na principu testovacího okna, ve kterém je zadaný text pro vyhledávání.
Pokud se v kódu nachází regulární výraz, tak se ve vedlejším okně zvýrazní části textu, kde vyhledávání proběhlo úspěšně. 

Mezi další nástroje, které existují pro zvolené prostředí, stojí za zmínku například \textit{Visual Regex} společně s \textit{Regexper unofficial}.
Tyto nástroje vygenerují obrázek NKA, který následně uživateli zobrazí.
Nedostatkem těchto nástrojů, je poměrně malá uživatelská interaktivita a flexibilita.

Existují další přístupy řešení pro VSCode. 
V případě mé aplikace, jsem podobný přístup pro dané vývojové prostředí jsem nenašel.
\chapter{Aplikační architektura}\label{sec:Principle}

\section{Využíté technologie}
Tato aplikace je integrovaná do vývojového prostředí \textbf{visual studio code}, 
zkráceně \textbf{vscode}. Jádro aplikace je psáno v programovacím jazyce \textbf{TypeScript} verze 5.3, který je nádstavbou
pro jazyk \textbf{JavaScript}. TypeScript, jak z názvu vyplívá je typový JavaScript.
Každý kód napsaný v JavaScriptu je správný pro TypeScript, ale to neplatí naopak.
Psaní nějaké větší aplikace je tak vhodnější v TypeScriptu, 
čímž se můžeme vyhnout potencionálním chybám v běhu programu.
Také vyvýjení rozšíření pro vscode, je možné pouze v JavaScriptu nebo TypeScriptu.

Pro parsování\footnote{proces kompilace a interpretace} je použita bezkontextová gramatika Peggy, pro jazyk JavaScript.
Ta umožňuje poměrně snadného zpracování textové podoby regulárních výrazů do podoby strukturované.
Výsledná struktura je ve formě NKA s AST (abstraktní syntaktický strom). 
NKA pak lze procházet a AST slouží k dohledání informací o syntaxi původního regulárního výrazu.


\chapter{Knihovna pro práci s regulárními výrazy}\label{sec:Implementation1}

Před tvorbou samotné vizualizace, jsem měl úvahu o použití knihovny, 
která by mi umožňovala získat zpracovanou strukturu historie průchodu regulárním výrazem.
Sice implementace regulárních výrazů se nachází v samotné specifikaci JavaScriptu,
ale ta mi neumožňuje získat informaci o samotném vyhledávání.
Po prozkoumání existujících řešení, která by vyhovovala této práci, 
jsem se rozhodl vytvořit vlastní implementaci v podobě této knihovny.
Nenalezl jsem totiž řešení, které by bylo dostatečně flexibilní a 
zároveň lehce integrovatelné do programovacího jazyka TypeScript.
Sice vlastní implementace může být pracná, ale jelikož chci mít co nejvyšší kontrolu nad výslednou strukturou, 
tak jsem toto řešení vyhodnotil za vhodné. 

V této části práce se pokusím vysvětlit, návrh této knihovny~\ref{sec:Imp1LayoutReal}, 
princip parsování regulárních výrazů~\ref{sec:Parse}, jak funguje vyhledávání a jak vypadá jeho výsledná struktura \ref{sec:PatternMatching}.

\section{Rozvržení}\label{sec:Imp1LayoutReal}

Vstupní třídou, pro tuto knihovnu je \textbf{Regexer}. 
Slouží jako spojující a zároveň obsluhující třída a také poskytuje rozhraní této knihovny.
Dále si drží důležité informace, které souvisí s aktuálně zpracovaným regulárním výrazem.
\textit{RegexMatch} je třída, která reprezentuje jeden výsledek vyhledávání zadaným výrazem.
Její data jsou soukromá, ale umožňuje je procházet pomocí svých metod.
Jedna instance teto třídy, je ekvivalentní jednomu vyhledání v textu.
Data této třídy jsou generována třídou \textit{MatchBuilder}.
Její instance existuje pouze ve chvíli, kdy probíhá vyhledávání v zadaném řetězci.
Poskytuje rozhraní, které umožňuje přidávat stavy, s tím že může tato data upravovat, dle své potřeby.
Tento objekt je pak vlastněný ve třídě \textit{MatcherInternal}, která má za úkol průchod zadaným řetězcem, pro konkrétní výraz.
Tato třída, je izolována a není dostupná z vnější, jak její název \textit{internal} (česky vnitřní) napovídá.
Obsahuje hlavní algoritmickou část průchodu nedeterministickým automatem.
Naopak třídou, která poskytuje viditelné rozhraní a volá metody třídy MatcherInternal je \textit{Matcher}.
Její rozhraní je poskytováno třídě Regexer.
Pro parsování textové reprezentace regulárního výrazu na strukturu NKA a AST, slouží rozhraní \textit{RegexParser}.
Jedná se o parser, který vzniká vždy po překladu bezkontextové gramatiky.
Stack je jednoduchá struktura, která je ale velmi důležitou součástí vyhledávání.
Zásobník programu umožňuje zbavit se rekurzivního volání funkce.
Rekurze obecně vede k pomalejšímu chodu programu a nelze ji jednoduše pozastavit v jakémkoliv čase a stavu.
Také může jednoduše při složitějším zpracování dojít k přetečení zásobníku, který je často limitován aby nedošlo k nekonečnému rekurzivnímu volání.
Sice rozhraní pole v JS je připraveno na funkcionalitu zásobníku, ale nezaručuje programátorovi striktní pravidla pro zásobník. 
Z tohoto důvodu jsem zvolil jednoduchou implementaci zásobníku, která omezuje manipulaci s základním polem, na operace určené pro zásobník.

Vztahy mezi jednotlivými třídami, lze vidět na obrázku~\ref{fig:ARCH_RGXR}. 
Nachází se zde také obalující blok \textbf{MatchingWorker}, který reprezentuje vstupní soubor vedlejšího vlákna, jež slouží pro asynchronní komunikaci s hlavním vláknem.
\textit{Regexer} jako vstupní třída, uchovává informace o AST a NKA regulárního výrazu.
Poskytuje také 2 hlavní metody, a to parse pro zpracování výrazu a match pro vyhledání v textu.
Po zavolání metody match, se zavolá metoda třídy \textit{Matcher}, pro zpracování výrazu na straně vlákna.

\begin{figure}[!h]
	\centering
	\includegraphics[width=0.9\textwidth]{Figures/UML_RGXR.pdf}
	\caption{Třídní diagram části knihovny pro práci s regulárními výrazy}
	\label{fig:ARCH_RGXR}
\end{figure} 

\section{Parsování regulárních výrazů}\label{sec:Parse}

\subsection*{Parser}

Jak již bylo zmíněno, pro parsování regulárních výrazu jsem použil bezkontextovou gramatiku \textit{Peggy}.
Jedná se o pokračování projektu PegJS, ale ten se již dlouho nevyvíjí. 
Jelikož knihovna Peggy je stále aktualizována a má velkou podporu vývojářů, tak jsem zvolil její využití pro tuto práci.

V ukázce~\ref{code:grammar1} se nachází vstupní neterminál bezkontextové gramatiky. 
Ten obsahuje výběr mezi dvěma začátky \textit{moded\_start} a \textit{general\_start}.
Výběr je pak dostupný pod názvem \textit{type}, podle toho který se zvolí.
Před dokončením pravidla, se na jeho konci může nacházet blok "\{\}", který může modifikovat výsledná data.
Modifikace v rámci pravidla pro začátek, probíhá zavoláním instance vlastní třídy \textbf{ParserHandler}, která je součástí gramatiky.
Třída má za úkol zpracovávat příchozí data do JSON struktury, která je ukázaná na obrázku~\ref{fig:JSONex}.

\begin{code}[!ht]
	\begin{minted}{peg}
start 
	= 
	type:(moded_start / general_start)
	{
		const data = { modifiers: type?.modifiers };
		return handler.handle(data, type?.elements, States.ROOT);
	}
	\end{minted}
	\caption{Počáteční neterminál}
	\label{code:grammar1}
\end{code}

Existuje několik různých vzorů regulárních výrazů.
Každý ze vzorů má pravidla, pro možné kombinování s ostatními vzory.
Ukázka kódu~\ref{code:grammar2}, obsahuje příklad možných výběrů pravidel, které lze společně kombinovat.
Například možnosti pro iteraci, ve zmíněném kódu \textit{to\_iterate}, obsahují pouze následující vzory, které mohou být opakovány.

\begin{itemize}
	\item Speciální znaky (\textit{escaped\_special}) -- "$\textbackslash s$", "$\textbackslash d$"
	\item Základní znaky (\textit{primitive}) -- "$a$", "$b$", "$0$"
	\item Výběr jakéhokoliv znaku (\textit{any\_character}) -- "$.$"
	\item Skupina (\textit{group}) -- "$()$"
	\item List znaků (\textit{list}) -- "$[a-z]$"
\end{itemize}

V mnoha případech záleží na pořadí výběru z dostupných vzorů, proto je potřeba určit, které možnosti upřednostnit.
Abych vysvětlil, proč je pořadí důležité vybral jsem si jako příklad \textbf{iteraci} (iteration) a \textbf{výběr} (option).
Zjednodušeně výběr má vyšší přednost, jelikož může mít za potomka iteraci.
Kdyby se neterminál iterace nacházel před neterminálem výběru, tak by došlo k tomu že by iterace nebyla součástí výběru, v případě že by se nacházela na pozici první možnosti výběru. 
Nebo-li byla by dříve zpracována, než-li samotný výběr.
Jako příklad může sloužit výraz $a*|b*$, při kterém by se první zpracovala iterace $a*$.
Výsledkem výběru by byly 2 možnosti $\epsilon$ nebo $b*$, což je sice sám o sobě správný tvar výběru, ale ve zvoleném výrazu \textbf{musí} být výsledný výběr $a*$ nebo $b*$.

\begin{code}[!ht]
	\begin{minted}{peg}
any_element 
	= option / iteration / optional / general

to_iterate
	= escaped_special / primitive / any_character / group / list
	\end{minted}
	\caption{Výběry neterminálů, pro některé vzory regulárních výrazů}
	\label{code:grammar2}
\end{code}

\subsection*{Struktura zpracovaného regulárního výrazu}

Pro zpracovaný regulární výraz, jsem zvolil strukturu, která obsahuje \textbf{nedeterministický konečný automat} (NKA), zároveň s \textbf{abstraktním syntaktickým stromem} (AST).
AST pak slouží k dohledání informací o původním regulárním výrazu. 
Výsledná struktura je datového typu, JSON (JavaScript Object Notation).
JSON strukturu je možné vidět na obrázku~\ref{fig:JSONex}.
NKA je ve formě \textbf{přesunové tabulky (transition table)}. 
Ta má tvar pole, kde každá položka obsahuje informaci o konkrétním stavu a přesunech na další stavy.
Stav je identifikovaný na základě indexu v poli. 
Přesuny jsou pak implementovány tak, že každý stav si uchovává všechny své přesuny, které vedou z daného stavu do stavu jiného.
Každý přesun pak má informaci, o jaký znak přesunu se jedná a na jaký index (stav) v poli odkazuje. 

Na obrázku~\ref{fig:JSONex} lze vidět základní JSON strukturu, která obsahuje dvě vstupní hodnoty AST a NFA.
Klíč NFA odkazuje na pole stavů přesunové tabulky. 
AST má odkaz na počátek struktury, který signalizuje začátek regulárního výrazu.
Je zde patrné, že každý stav v tabulce přesunů, má odkaz na příslušící prvek v AST. 
AST prvek/vzor drží různé informace, např. pozice v původním řetězci (start a end), 
potomky daného stavu, nebo typ vzoru. 
Ne každý stav musí mít potomky, ale například skupina potomky má.
Některé vzory, jako je například iterace obsahují v AST dodatečné informace, jako je rozsah iterace (range).
Typy stavů jsou číselné hodnoty, a ty které se nacházejí na obrázku jsou zde také popsány.

\begin{figure}[!h]
	\centering
	\includegraphics[width=1\textwidth]{Figures/BP-JSON.pdf}
	\caption{Příklad výsledné JSON struktury regulárního výrazu a+}
	\label{fig:JSONex}
\end{figure}

\newpage

\section{Vyhledání pomocí regulárního výrazu}\label{sec:PatternMatching}

Vyhledání je jednou z hlavních částí této knihovny, jedná se o procházení regulárním výrazem a hledaným řetězcem.
Výsledkem je struktura JSON dat, která obsahuje informace o zpracovaném vyhledávání.
V této části textu popisuji, jak jsem na implementoval vyhledávání, důležité koncepce a výslednou strukturu.

\subsection*{Odstranění rekurze}

Rekurze je sice důležitým aspektem mnoha programů a dokáže usnadnit některé problémy, ale existují situace kdy se vyplatí jí zbavit.
V první řadě, bych rád vysvětlil, proč je rekurzivní řešení vhodné pro vyhodnocování regulárních výrazů.
Jak jsem již zmínil, tak v regulárním výrazu může dojít k backtrackingu.
Nastane ve chvíli, kdy není možné z daného stavu v NKA, přejít na stav jiný.
V tuto chvíli, dojde k vrácení se v NKA do předchozího stavu a k pokračování vyhledávání pomocí další možné cesty.
Nejjednodušší řešení tohoto případu, je použití rekurze.
Pro představu, přechod značí rekurzivní volání funkce a pokud není možné přejít do dalšího stavu, tak se vrací do předchozího volání funkce.

Rekurzi lze odstranit pomocí zásobníků, nebo-li program si uloží, jen potřebné informace.
Ve chvíli kdy dojde k vrácení se (backtrackingu), tak se odstraní vrch zásobníků.
Důležité tedy je správně řídit správu zásobníků, což může být lehce komplikované.

Úryvek zdrojového kódu~\ref{code:matching1}, obsahuje základní vkládání nově navštíveného stavu do zásobníku.
Přesněji se zde ukládá jak stav, tak také index 0 značící počáteční přesun.
Pokud se stav již nachází na vrchu zásobníku, tak je pouze navýšen index přesunu.

% code pushing new state to stack
\begin{code}[!ht]
	\begin{minted}{typescript}
const nfaState = NFA[<number>this.regexPosStack_.top()] as NFAtype;
let topState = this.statesStack_.top();
if(topState?.state !== nfaState)
	this.statesStack_.push({transition: 0, state: nfaState});
else
	topState.transition++;
	\end{minted}
	\caption{Vložení stavu do zásobníku}
	\label{code:matching1}
\end{code}

Příklad kódu~\ref{code:matching2}, souvisí s předchozí ukázkou. 
Jestli nastane situace, kdy neexistuje žádný další přesun z aktuálního stavu, je vyvolán backtracking.
Metoda \textit{handleBacktracking} se stará o správu backtrackingu.
Převážně se jedná o odebírání vrchů zásobníků, jako je již zmíněný zásobník stavů.
Pokud není vrácená hodnota \textbf{null}, znamená to ukončení nebo pozastavení vyhledání.
K neúspěšnému vyhledání a následnému ukončení dojde, pokud je zásobník stavů vyprázdněný a pozice v hledaném řetězci se nachází na jeho konci.

% code poping state from stack
\begin{code}[!ht]
	\begin{minted}{typescript}
const transitions = (nfaState as NFAState).transitions;
if(transitions.length <= <number>this.statesStack_.top()?.transition)
{
	const returned = this.handleBacktracking();
	if(returned !== null) return returned;
	continue;
}
	\end{minted}
	\caption{Vyvolání backtrackingu, pokud neexistují další přechody ze současného stavu}
	\label{code:matching2}
\end{code}

\newpage

\subsection*{Počítání iterací a prevence nekonečných cyklů}

\begin{figure}[!h]
	\centering
	\includegraphics[width=0.65\textwidth]{Figures/IterationCount.pdf}
	\caption{NKA pro popis počítání iterací a prevenci nekonečných cyklů}
	\label{fig:ITERCNT}
\end{figure}

V seznamu pravidel, $I_i$ značí identifikátor iterace, $i$ je počet dokončených opakování iterace a $P_s$ je aktuální pozice v hledaném řetězci.
$I_{arr}$ obsahuje informace v poli o jedné iteraci [$I_i$, $i$, $P_s$].

Algoritmus pracuje se dvěma zásobníky, určenými pro držení informací o iteracích. 
První uchovává aktuálně nedokončené, resp. probíhající iterace. 
Pro popis jej označuji $Z_s$ (zásobník současných iterací).
Druhý značím jako $Z_h$ (zásobník historie), ten slouží pro iterace, které byly již dokončené.
Historie je důležitá pro backtracking.

\begin{enumerate}[label=\arabic* --]
	\item Vlož do $Z_s$ $\longleftarrow$ [$I_i$, 0, $P_s$]
	\item Vlož do $Z_h$ $\longleftarrow$ [$I_i$, 0, $P_s$]
	\item Vrchol $Z_s$, $I_{arr}$[1] + 1
	\item Vlož do $Z_h$ $\longleftarrow$ Odeber ze $Z_s$, $I_{arr}$[1] + 1
	\item Odeber ze $Z_s$
	\item Odeber ze $Z_h$
	\item Vrchol $Z_s$, $I_{arr}$[1] - 1
	\item Vlož do $Z_s$ $\longleftarrow$ Odeber ze $Z_h$, $I_{arr}$[1] - 1
\end{enumerate}

Na obrázku~\ref{fig:ITERCNT} lze vidět nedeterministický automat. 
Jedná se o obecnou reprezentaci iterace, kde $z_i$ reprezentuje začátek iterace, $k_i$ konec iterace a $q$ značí první stav za iterací.
Mezi $z_i$ a $k_i$ se nachází množina stavů $n(s)$.
Červené šipky signalizují backtracking a černé značí klasický přechod mezi stavy.

Jelikož existují iterace v rozmezí, například od 3 do 6, tak je potřeba znát informaci, v kolikátém opakování se právě konkrétní iterace nachází.
Pro tento problém jsem zvolil 8 pravidel, které popisují řešení osmi různých přesunů mezi stavy.
Tato pravidla jsou rozepsána pod obrázkem~\ref{fig:ITERCNT}, indexy pravidel korespondují s indexy v obrázku.
Rád bych poukázal na to, že při backtrackingu se vždy zásobníky vrací, do původních stavů. 
To znamená, mám-li například stavy $a$ a $b$, tak platí pro přesun $a \rightarrow b$ a pro následující backtracking $b \rightarrow a$, že ve stavu $a$ musí být po dokončení obou přechodů hodnoty zásobníků ve stejném stavu jako na začátku.
Jedno opakování je dokončeno při přechodu 3 nebo 4. 
Zároveň přechod 4 společně s přechodem 2, jsou konečnými přechody pro danou iteraci.

Zásobník současných iterací obsahuje poměrně malé množství informací, jelikož se jedná pouze o probíhající iterace.
Naopak zásobník historie může obsahovat poměrně hodně informací. 
Má-li iterace například 100 dokončených opakování, tak historie bude obsahovat minimálně 100 záznamů.
To se může zdát jako mnoho zbytečných informací, ale nelze předem prakticky vědět jestli dojde k backtrackingu a kde se zastaví.

Další důležitou kontrolou, kterou je nutné splnit, je na konci iterace zkontrolovat zda se nachází v určeném rozmezí.
Jelikož počítám jejich opakování, tak stačí tuto informaci porovnat s náležitými mezemi.

V některých případech by mohlo dojít k nekonečnému cyklu.
Například pro regulární výraz $()+$ by k tomu došlo tak, že by nenastalo k posunu v hledaném řetězci, ale přes to by stále procházela úspěšně dále.
K tomu slouží ukládání poslední pozice v hledaném řetězci, při začátku nové iterace, nebo zopakování.
Jestli má dojít k zopakování, musí proběhnou kontrola, zda-li došlo ke změně pozice v řetězci od posledního zopakování.
V obrázku~\ref{fig:ITERCNT} se jedná o stav $k_i$ a přesun $3$.

\subsection*{Využití vlákna pro vyhledávání}
Nedílnou součástí této knihovny je \textbf{paralelní zpracování} v podobě balíčku threads.js.
Balíček byl již zmíněn v kapitole~\ref{sec:USEDtech}.
Pralelismus dovoluje složité operace přesunout do vedlejšího vlákna, aby hlavní vlákno nebylo zatěžováno.
Vlákna sice umožňují efektivnější zpracování náročných programů, ale také mají svá úskalí.

S volbou vývojového prostředí \textit{VSCode}, byla nutnost splnit podmínky stanovené pro práci s web workery, v souladu s jejich API \cite{Microsoft_2021}. 
Podmínkou totiž je, nutnost mít zdrojový kód workeru přímo vložený ve zdrojovém kódu hlavního vlákna.
To znamená, že worker nesmí být přímo načítaný, z adresáře rozšíření.
Avšak tato nutnost, je komplikovaná a proto následovně vysvětlím, jak jsem tento problém řešil.

Všechny závislosti, které worker má, musí být součástí jednoho výsledného souboru.
To je docíleno tím, že přeložím soubor pomocí \textit{webpack}, který vytvoří jeden výsledný soubor.
Pokud by někdo chtěl využít této knihovny, v rámci prostředí NodeJS nebo Prohlížeče, tak tento překlad probíhá dvakrát pro oba runtime.
Tento soubor může následně být vložen přímo do zdrojového kódu.
Pokud aplikace, která využívá tuto knihovnu má webpack, může využít loaderu, který jsem pro tuto knihovnu napsal. 
Ten dokáže v místě kde je worker volaný, vložit jeho zdrojový kód, v rámci textového řetězce.
Výsledkem je worker, který je vložený jako řetězec, ve zdrojovém kódu hlavního vlákna.

Původně tato knihovna zprostředkovávala, pouze paralelismus pro prostředí \textit{NodeJS}.
Později se ale ukázalo, že tato restrikce je limitující, co se týče využití této knihovny ve vizualizační části aplikace.
Regulární výrazy se totiž zpracovávali, na straně rozšíření, které běží v prostředí \textit{NodeJS}.
Vizualizace byla tedy omezena na komunikaci s prostředím, co se týče práce s regulárními výrazy.
Vývojové prostředí pak sloužilo, jako komunikační uzel mezi vizualizací a touto komponentou.
Také toto zprostředkování, poměrně zpomalovalo výkon aplikace, jelikož se musela data posílat pomocí zpráv mezi dvěma komponentami.
Samotná vizualizace, nemohla existovat jako samostatná webová stránka, jelikož zde existovala přímá závislost na rozšíření.
Změnou této části aplikace, která současně podporuje využití paralelismu, pro jakákoliv prostředí, byly tyto problémy eliminovány.

\subsection*{Výsledek vyhledávání}

Výsledkem vyhledávání je třída, obsahující data s informacemi o procházení.
Jejich tvar se neřídí žádným standardem, nebo-li výsledná struktura je čistě přizpůsobená této práci.
Vlastnosti výsledného objektu obsahují všechny důležité informace.
První hodnotou je, zda-li bylo vyhledání úspěšné, či nikoliv. 
Další jsou skupiny, které drží informace, kde se nachází v regulární výrazu a hledaném řetězci. 
Pokud se jedná o pojmenovanou skupinu, tak se také ukládá její jméno. 
Poslední vlastností, která stojí za zmínku je pole, nebo-li seznam všech po sobě jdoucích stavů.

Ve stavech se nachází údaje, které reprezentují historii průchodu.
Každý stav obsahuje, údaj o pozici v řetězci a ve výrazu. 
Také musí být identifikován, o jaký vzor regulárního výrazu se jedná.
Stav může obsahovat další data, která jsou nepovinná, nebo-li se nenachází ve všech stavech.
Jedná se převážně o typ akce a seznam skupin.
Akce je informace, která upřesňuje typ stavu, jako je například backtracking.
Seznam skupin se může nacházet, také v jednotlivých stavech.
Lze pak pozorovat průběh vývoje skupin, s vývojem stavů.

Výsledné stavy se mohou lišit, jak dle počtu, nebo také podle tvaru. 
Modifikace vznikne na základě předem určených nastavení.
Ta například umožňují zahodit nežádoucí informace, nebo naopak přidat rozšiřující.
Zvolil jsem tuto možnost nastavení, aby knihovna mohla být univerzálnější a flexibilnější.

Data jsou uložená v JSON struktuře. 
Ta je dále součástí třídy \textbf{RegexMatch}.
Samotná třída poskytuje pouze rozhraní pro procházení stavů, nebo popřípadě pro získání dalších základních informací o vyhledávání.

\endinput
\chapter{Vizualizační knihovna a rozšíření}\label{sec:Implementation2}

Tato část aplikace se zabývá, knihovnou určenou pro vizualizaci průchodu regulárním výrazem a rozšířením do vývojového prostředí VSCode.
Pro získání informací o vyhledávání, slouží již zmíněná knihovna, která byla popsána v předchozí kapitole \ref{sec:Implementation1}.
Hlavním cílem této části aplikace je, na implementovat uživatelsky přívětivé a intuitivní rozhraní, integrované do vývojového prostředí.
To umožňuje zadávat regulární výrazy, text ve kterém lze pomocí zadaného výrazu vyhledávat a následnou vizualizaci ve formě debuggeru.
Vizualizace je koncipována, jako webová stránka, která je zobrazená do prostředí pomocí tzv. webview.
%TODO

\section{Návrh vizualizační části}

Vstupem knihovny je HTML soubor \textit{main.html}.
Ten vkládá skript \textit{index.ts}, jehož hlavním účelem je inicializovat vše potřebné pro chod aplikace.
Hlavní třídou, která se stará o obsluhu vizualizace je RegexVisualizer. 
Jejím úkolem je obsluhovat ostatní komponenty a přímo komunikuje s třídou \textit{Regexer} a tím získává data pro vizualizaci.
Pro interakci uživatele slouží dva textové editory, ty jsou ve formě dvou tříd \textit{RegexEditor} a \textit{StringMatchEditor}.
Obsluhují HTML elementy pro zadávání textu, jejich základní funkcionalita je děděna ze třídy \textit{TextEditor}.
Pro samotnou vizualizaci ve formě ladícího nástroje, existuje třída \textit{RegexDebugger}.
Má za úkol, obsluhovat okno aplikace, kde se samotný debugger nachází.
Obsahuje i vlastní posuvník (\textit{Slider}), který dokáže vyvolávat události, s informací o aktuální hodnotě posuvníku.
Také má například možnost automatického přehrávání a změnu rychlosti.

%TODO

\section{Implementace textových editorů}

Ve vizualizaci se nachází dva textové editory.
První pro zadávání regulárního výrazu a druhý pro zadávání řetězce, ve kterém proběhne vyhledávání, na základě zadaného výrazu.
Jejich funkcionalita je obalena ve dvou třídách, každá sloužící pro jiný textový editor a obě dědí ze třídy \textit{TextEditor}.
Tento vztah lze vidět na obrázku \ref{fig:TextEditor}.

\begin{figure}[!h]
	\centering
	\includegraphics[width=0.8\textwidth]{Figures/TextEditor.pdf}
	\caption{Třídní diagram textových editorů}
	\label{fig:TextEditor}
\end{figure} 

\subsection*{Základní řešení}

Pro řešení textových editorů, jsem se rozhodl pro vlastní implementaci, pro větší flexibilitu a řešení konkrétních problému, týkajících se práce s regulárními výrazy.
Textový editor umožňuje rozšířené možnosti práce s textem, oproti HTML elementům, jako jsou input nebo textarea.
Tyto možnosti jsou například vlastní formátování, nebo správa historie. 
K realizaci samotných vstupů, jsem použil základní obalující HTML blok span.
Na zvoleném bloku tolik nezáleží, ale je potřeba, aby měl atribut \textit{contenteditable}, s nastavenou hodnotou na true.
Tento atribut povoluje psaní přímo do daného bloku. 
Oproti elementu jako je input, lze zde vkládat HTML kód a tím upravovat formátování textu.
To je vhodné pro regulární výrazy, jelikož sami o sobě nejsou moc přehledné.

\textit{TextEditor} slouží jako vzorová třída, pro realizaci textových editorů.
Drží si referenci na HTML element, který obsluhuje, pod názvem \textit{textInput\_}. 
Pro interaktivitu s tímto elementem, je potřeba zaregistrovat různé události.
Mezi ně patří např. psaní, mazání, undo a redo. 
Události jsou registrovány při vytvoření instance třídy, pomocí soukromé metody \textit{registerEvents}.
Pokud je zavolána, chráněná metoda \textit{handleFormating}, tak dojde ke změně podoby textu na formátovanou.
Jedná se o grafické zobrazení bílých znaků, jako je nový řádek nebo tabulátor.

%TODO
% TODO: mby add element helper

\subsection*{Komunikace s třídou Regexer}

Komunikace funguje pomocí vyvolání událostí z textových editorů.
Je-li událost vyvolána editorem pro regulární výraz, tak se pomocí třídy \textit{Regexer} parsne daný výraz.
Pokud dojde k vyvolání události z editoru pro hledaný řetězec, tak dojde k zavolání metody pro vyhledávání (match).

Vyhledávání pak funguje na bázi asynchronní komunikace, kdy se výsledky posílají po tzv. dávkách (anglicky batch).
Výhodou této komunikace je, pokud uživatel v průběhu zpracovávání změní text jednoho z polí, tak se proces ukončí a tím pádem nemusí čekat na jeho dokončení.
Velikost jedné dávky jsem zvolil na 20000 stavů, jelikož tento počet stavů, je poměrně rychle vyhodnocený.
Informace o zpracování, se aktualizují průběžně po každé příchozí dávce.

%TODO

\subsection*{Zvýraznění syntaxe}

Součástí třídy \textit{RegexEditor}, je metoda sloužící pro zvýrazňování syntaxe regulárních výrazů. 
Pro zvýraznění slouží získaná AST struktura po dokončeném parsování.
Algoritmus řešení tohoto problému, funguje na principu rekurzivního zanoření, ve stromové struktuře.
Každý symbol, který má být zvýrazněný, je obalený v HTML bloku, s třídou identifikující o jaký symbol se jedná.
U jednotlivých vzorů, záleží na pořadí zpracování symbolů a rekurzivního zanoření.
Například skupina, se zpracovává tak, že první se zvýrazní otevírací závorka "(".
Poté se algoritmus rekurzivně zanoří, nebo-li zpracuje potomky (vnitřní část) skupiny a nakonec zvýrazní ukončující závorku ")".
Výsledkem vznikne HTML struktura, která popisuje symboly a vzory regulárních výrazů.
Tyto symboly jsou pak zvýrazněny, pomocí různých barev, definovaných v CSS.

\subsection*{Historie}
Práci s historii jsem musel na implementovat vlastní, jelikož původní nefungovala správně.
Důvodem bylo časté přepisování textu, z již zmíněného formátování.
Aby historie fungovala, musel jsem vytvořit pole, které obsahuje jak původní řetězec tak pozici kurzoru v něm.
Pokud byla vyvolána operace vrácení se zpět v historii (undo), překopíruje se uložený řetězec do textového pole a kurzor se nastaví na správnou pozici.
K odstranění záznamu z historie nedochází, jelikož může být vyvolána \textit{redo} operace, nebo-li odvolání operace \textit{undo}.
Pokud dojde k uložení nového stavu textového pole, tak všechny stavy za ukazatelem se smažou a přidá se zde nový.
V nastavení editoru, jsem přidal možnost zvolit si maximální počet záznamů v historii.
Pokud ale není nastavená, automaticky se omezí na 100 záznamů.

\subsection*{Pozice textového kurzoru}
Práce s textovým kurzorem je další značná část textových editorů.
Pokud uživatel píše do textového pole, často nastává ke změně textu na pozadí samotnou aplikací.
Například při zvýraznění, dochází ke změně textové formy na HTML strukturu.
Při změně vždy dojde k resetování pozice kurzoru v textu.
To ale pro uživatele není příjemná vlastnost, kterou jsem tedy musel vyřešit.

Před přepsáním textového pole, je uložená pozice kurzoru.
Po vložení textu, je nutné vrátit se na uloženou pozici. 
Nejedná se ale o jednoduchou úlohu, jelikož pokud se v textu nachází HTML elementy, musí být brány v potaz.
Použil jsem základ algoritmu ze stránky \textit{stackoverflow}\footnote{https://stackoverflow.com/questions/69956977}, který dokáže jak zjistit aktuální pozici kurzoru, tak z pozice umístit kurzor na správné místo.
Ten jsem upravil pro potřeby mého projektu a dále rozšířil.
Například jsem přidal možnost vytvoření nového kurzoru, který není přímo vložený do dokumentu, což se může hodit pokud je potřeba získat souřadnice písmena.

\section{Vizualizace průchodu}

Vizualizace ve formě debuggeru je obsluhována třídou \textit{RegexDebugger}.
Okno pro vizualizaci se otevře po kliknutí na tlačítko, které je předáno třídě součástí konstruktoru.
Debugger obsahuje identická pole jako textová pole pro interakci s uživatelem, avšak již nelze jejich text editovat.
Dále disponuje posuvníkem, který slouží pro procházení průběhu vyhledávání. 

\subsection*{Posuvník}

Posuvník jsem zvolil, jako jednoduchou a intuitivní možnost procházení historie.
Jeho implementace je ve vlastní třídě a jeho součástí je nastavení, pro příjemnější manipulaci výsledného posuvníku.
Pomocí nastavení, lze vypínat/zapínat některé funkcionality, nebo měnit samotný vzhled, jako je barva či velikost.
Tato realizace je vlastní, z důvodu lehčí integrace do aplikace.

Při vytváření instance této třídy, musí být předán HTML element nebo id elementu, do kterého se posuvník vykreslí.
Nastavení je dobrovolné, pokud není předáno zvolí se základní.
Posuvník může mít tlačítka pro ovládání, těmi jsou automatické přehrávání, vpřed, zpět, na konec a na začátek.
Pro automatické přehrávání, může být součástí pole, pro editaci rychlosti, pokud je povolené v nastavení.

Posuvník může nabývat pouze celočíselných hodnot, v omezeném rozmezí od \textit{min} do \textit{max}.
Pokud se změní jeho hodnota, je vyvolána vlastní událost, která tuto hodnotu obsahuje.
Ta může být odchycena, např. jinou třídou.

\subsection*{Zvýraznění pozice a backtrackingu}

Pro vizualizaci slouží zvýraznění pozice, jak v regulárním výrazu, tak v hledaném řetězci.
Tento koncept jsem převzal z inspirativní stránky \textit{regex101.com}.
Pozice je zvýrazněná tak, že je v pozadí pozice barevný blok, který je vykreslený do HTML plátna (canvas).
Řešení tohoto problému, jsem několikrát změnil, jelikož se původní řešení ukázalo neúčinným v některých případech.

Jako první řešení, jsem zvolil získání šířky písmene, výšky řádku a velikost mezery mezi písmeny.
Poté jsem procházel celí řetězec a pokud byl znak součástí pozice pro zvýraznění, tak jsem rozšířil šířku zvýrazněného bloku o šířku písmene a velikost mezery.
Jesli byl nalezen znak nového řádku, nebo délka zvýrazněného bloku přetekla velikost řádku, tak jsem vytvořil nový blok pro nový řádek.
Problém tohoto řešení byl takový, že když došlo k nekontrolovanému zalomení řádku tzn. pokud byl zalomen řádek dříve, než konec tohoto řádku.
To mohlo nastat, například pokud se zalomilo slovo.
Ve výsledku docházelo k zvýraznění prázdného místa a také k jeho nesprávnému konci.

Druhé řešení, které jsem zkusil na implementovat, bylo pomocí využití textového kurzoru.
Princip byl již jiný, jelikož nebylo třeba znát velikost písmene a mezery.
Kurzor jsem nejprve umístil, na začáteční pozici zvýraznění. 
V cyklu, jsem postupně posouval kurzorem až na konec zvýraznění.
Během tohoto procesu jsem si ukládal souřadnice, kde se nachází.
Tento způsob již zamezil problému při zalamování řádku, ale byl poměrně neefektivní a dokázal zpomalovat uživatelskou interaktivitu.

Poslední zpusob implementace, dokáže vyřešit i zmíněný problém se zpomalením.
Vychází z předchozího popisu, jelikož také využívá kurzor.
Rozdílem je, že kurzor je vložen jako rozsah od začátku až po konec zvýraznění.
Není tedy třeba procházet, písmeno po písmenu.
Kvůli toho jsem upravil kód, pro získání a nastavení pozice kurzoru, tak aby umožňoval také výběr.
Tato implementace se ukázala jako nejlepší, z důvodu výkonu a funkcionality.

Pokud je pozice délky nula, nebo-li \textit{min} je stejný jako \textit{max}, tak je stále zobrazena. 
Její šířka, je pak velikost mezery mezi dvěma písmeny. 
Dále jsem musel zohlednit, jestli text má scrollbar. 
Pokud ano, tak samotné zvýraznění musí být v plátně posunuto, o výšku aktuálního scrollu.

Backtracking je vyhodnocený stejnou funkcí, jako pro zvýraznění pozice.
Jediná věc, která se liší, je forma zobrazení. 
Ta je ve tvaru šipky, signalizující odkud a kam se přesouvá v regulárním výrazu.
Výška šipky není stejná, jako výška řádku, ale je zkrácená konkrétně na 2 pixely.

\subsection*{Zobrazení skupin}

Skupiny jsou podobně zobrazeny, jako pozice vyhledání, nebo-li ve formě zvýraznění části textu.
Jelikož skupiny mohou být vnořené, je třeba předem určit pořadí, ve kterém se budou zakreslovat.
Kdyby nebyly řazeny, tak by se mohlo stát, že vnější skupina přepíše vnitřní.
Pro samotné zobrazení, je potřeba měnit barvu každé skupiny, nebo zvolit jiný způsob rozlišení, aby je bylo možné rozeznat.
Zvolil jsem první způsob, kdy podle indexu skupiny je zvolená její barva.

\section{Uživatelské rozhraní}

Základní zobrazení, je ve tmavém režimu, které lze vidět na obrázku \ref{fig:GeneralUI}.
Rozhraní je koncipováno pouze na jednu stránku a obsahuje poměrně jednoduché ovládání.
Základem rozhraní jsou dvě textová pole, kde první slouží pro zadávání regulárních výrazů a druhé pro hledaný řetězec.
Oba vstupy aplikace automaticky vyhodnocuje, nicméně vstup pro hledaný řetězec, čeká nějakou dobu, než uživatel dopíše, aby častá aktualizace nezpomalovala aplikaci.

Po dokončeném zpracování, se v pravé spodní části aplikace nachází informace, které jsou vidět na obrázku \ref{fig:GeneralUI}.
Zobrazuje se zde, kolik úspěšných vyhledání se povedlo a v kolika krocích.
Také vedle zmíněných informací, je umístěna ikona signalizující informaci o průběhu zpracování.
Ikona může být zobrazena čtyřmi různými způsoby.
První je ukázán na obrázku a jedná se o celkový úspěch vyhledání.
Další 3 ikony značí neúspěch, načítání resp. probíhající zpracovávání a poslední čekání na správně zadaný regulární výraz.

V levé spodní části aplikace je tlačítko pro otevření debuggeru, po jeho otevření se zobrazí okno, které je ukázáno na obrázku \ref{fig:DebuggerUI}.
Na vrchu okna se nachází, posuvník který slouží pro průchod stavů.
S ním souvisí tři pole, které jsou pod posuvníkem.
Aktuální stav, nebo-li hodnota posuvníku, se nachází v levém poli.
Uprostřed je ovládání pomocí pěti tlačítek: začátek, zpět, automatické přehrávání, dopředu a konec.
Poslední políčkem souvisejícím s posuvníkem, je pro manipulaci rychlosti automatického přehrávání.
Rychlost je pak vyjádřena, jako $1/n$ stavů za sekundu, kde $n$ je nastavená rychlost, v případě obrázku \ref{fig:DebuggerUI} je $n = 5$.

Dále se v debuggeru nachází dvě pole, pro vizualizaci stavů regulárního výrazu a druhé pro vizualizaci pozice v hledaném řetězci.
Stavy jsou automaticky aktualizovány, po změně hodnoty posuvníku.
V regulárním výrazu v obrázku, je zrovna vyobrazen backtracking (červená šipka zpět).
V hledaném řetězci, se zvýrazněná pozice, aktuálního stavu vyhledávání.
Jeho součástí mohou být zobrazeny skupiny, pokud nějaké již byly dokončeny.

\begin{figure}[!h]
	\centering
	\includegraphics[width=0.85\textwidth]{Figures/appWindow.png}
	\caption{Úvodní uživatelské rozhraní}
	\label{fig:GeneralUI}
\end{figure}

\begin{figure}[!h]
	\centering
	\includegraphics[width=0.85\textwidth]{Figures/appDebugger.png}
	\caption{Uživatelské rozhraní debuggeru}
	\label{fig:DebuggerUI}
\end{figure}

VSCode api zpřístupňuje využití css stylů, které má uživatel přímo nastavené ve svém prostředí.
Pokud je tedy webová stránka součástí VSCode prostředí, tak přejímám styly, které má uživatel přímo nastavené.
Součástí toho jsou, fonty, barvy, tmavý nebo světlý režim atd.
Jestliže má uživatel nastavený světlý režim, tak se stránka automaticky přizpůsobí.

\section{VSCode rozšíření}

Samotné rozšíření je vyvíjeno, pro prostředí Visual Studio Code.
Výhodou tohoto prostředí, je lehká integrace webového rozhraní.
VSCode API, samozřejmě zprostředkovává mnohem více možností, pro samotný vývoj rozšíření.

\subsection*{Integrace webového zobrazení}

Webové zobrazení, využívá tzv. webview, které zobrazí webovou stránku v omezeném režimu.
Omezení je například ve smyslu, nutnosti explicitně povolit JavaScriptové kódy, spouštěné v samotném zobrazení.
Dalším problémem je načítání zdrojů.
Ty totiž nemohou být jednoduše brány z lokálních adresářů. 
Musí dojít ke konverzi adresy, na adresu podporovanou VSCode. 
K této změně, jsem použil vyhledání všech cest ke zdrojům.
Před tím než dojde k zobrazení webové stránky, je tato konverze provedena.

Zobrazení se může nacházet v několika částech prostředí.
Pro tuto aplikaci jsem zvolil, že se stránka zobrazí, v pravé části na polovinu obrazovky.
V budoucnu by, bylo možné přidat nastavení, aby si uživatel mohl zvolit, kde se bude okno aplikace zobrazovat.

Zobrazení aplikace lze také vyvolat, najetím v textu na zadaný regulární výraz.
Po najetí se zobrazí tlačítko, které po jeho stisku otevře samotnou vizualizaci.
V té se bude již nacházet daný výraz, na který uživatel najel v textu.
Jelikož je aplikace integrována ve vývojovém prostředí, tak je podle mě výhodné mít jako přidanou hodnotu, využívání větší interaktivity s prostředím.
VSCode API poskytuje, možnost detekce pozice kurzoru myši, čehož jsem využil pro následnou detekci.
Aby bylo možné rozpoznat, zda se na pozici v textu nachází syntaxe regulárního výrazu, musí proběhnout nějaké vyhodnocení textu.
Toto vyhodnocení probíhá pomocí specifického regulárního výrazu, který zjišťuje zda se jedná o správný tvar.
Zmíněné řešení mě napadlo jako první, ale není ideální.
Vhodnější by bylo použít nějaký jazykový server, který by z daného textu automaticky rozpoznal, o jaký syntaktický prvek se jedná.
%TODO

\subsection*{Možnosti nastavení rozšíření}

API pro VSCode nabízí možnost poskytnout nastavení, pro rozšíření, které může uživatel měnit na základě svých preferencí.
Pro tuto aplikaci jsem přidal jednoduché a omezené možnosti nastavení.
První se týká možnosti, vypnutí a zapnutí funkce, pro najetí na text zmíněné v předchozí sekci.
Druhé nastavení udává, zda se má vytvářet pokaždé nové okno webového rozhraní, nebo se má překreslovat již existující.
Nastavení by mohlo existovat v budoucnu více, aby umožňovali uživatelům vyšší flexibilitu. 

%TODO

\subsection*{Komunikace rozšíření a vizualizace}

Ke komunikaci VSCode s webview dochází pomocí zpráv.
Jedná se o komunikaci, která je předepsána pro VSCode API, v jejich dokumentaci.
Komunikaci jsem se pokusil omezit na minimum, aby nedocházelo k přetížení aplikace.
Proto se posílají pouze zprávy, které souvisí s samotným rozšířením pro VSCode.
Například pokud dojde, k otevření výrazu z VSCode, tak musí dojít k jeho poslání pomoci zprávy do webview.
Vizualizace se také může nacházet samostatně, mimo kontext rozšíření, proto je posílání a přijímání zpráv vypnuto.
%TODO

\subsection*{Výsledný instalační soubor}

Výsledným souborem, pro VSCode rozšíření, je soubor s příponou \textit{.vsix}.
Tyto soubory slouží pro lokální instalaci rozšíření. 
Samotná rozšíření lze publikovat veřejně součástí VSCode, ale aktuálně jsem se rozhodl, nepublikoval tuto aplikaci.
Pro získání \textit{vsix} souboru, slouží příkaz \textit{vsce}, který dokáže sjednotit všechny závislosti do jednoho výsledného souboru.

Pro instalaci rozšíření, lze použít příkazovou paletu od VSCode.
Tu lze otevřít pomocí zkratky \textbf{Shift + Command + P} pro Mac a \textbf{Ctrl + Shift + P} pro Windows nebo Linux.
Po otevření příkazové palety, stačí zadat příkaz \textit{Extensions: Install from VSIX}.
Poté se otevře průzkumník souborů, kde stačí otevřít zvolený \textit{vsix} soubor, který se nainstaluje.

%TODO

\section{Sestavení a spuštění projektu}

%TODO

\endinput
\chapter{Zhodnocení a testování výsledků}\label{sec:Testing}

Pro testování při vývoji, jsem používal technologii Jest, pro psaní vlastních testů.
To hlavně kvůli udržitelnosti kódu a včasnému zachycení chyb při vývoji.
Testy jsem používal pouze pro část aplikace, která používá hodně algoritmického kódu.
Konkrétně při vývoji knihovny pro práci s regulárními výrazy \ref{sec:Implementation1}.
Testy jsem psal postupně podle případů, které mě napadly při vývoji v určitých částech vývoje.
Tento přístup byl samozřejmě nedostačující a tak při testování samotné vizualizace, se mi povedlo najít několik dalších případů špatných výsledků. 
Ty jsem následně opravil a přidal do samotných testů.

Testování proběhlo na počítači s 6 jádrovým procesorem a 12 vlákny, AMD Ryzen\texttrademark ~5 5600X.
Dostupná paměť na použitém stroji, je 32GB RAM.

\begin{table}[!h]
	\centering
    \begin{tabular}{ |c|c|c|c|c|c|c|c|c| }
        \cline{2-9}
        \multicolumn{1}{c|}{} & \multicolumn{8}{c|}{\textbf{Regulární výraz a testovací řetězec}} \\
        \multicolumn{1}{c|}{} & \multicolumn{8}{c|}{Čas v ms} \\
        \hline
        \textbf{Testovací subjekt} & \ref{itm:TD_1a} & \ref{itm:TD_2a} & \ref{itm:TD_3a} & \ref{itm:TD_4a} & \ref{itm:TD_5a} & \ref{itm:TD_6a} & \ref{itm:TD_7a} & \ref{itm:TD_7b}  \\
        \hlineB{3}
        Regexer (vlastní knihovna) & 1.936 & 0.672 & 1.061 & 34.205 & 1.004 & 1.556 & 2.382 & 116.117 \\
        \hline
        RegExp & 0.039 & 0.023 & 0.031 & 0.107 & 0.051 & 0.045 & 0.053 & 0.029 \\
        \hline
        match & 0.037 & 0.024 & 0.028 & 0.034 & 0.032 & 0.035 & 0.035 & 0.025 \\
        \hline\hline
        \rule{0pt}{14pt} \textbf{Nalezený výsledek} & \textcolor{OliveGreen}{\Checkmark} & \textcolor{OliveGreen}{\Checkmark} & \textcolor{OliveGreen}{\Checkmark} & \textcolor{Red}{\XSolid} & \textcolor{OliveGreen}{\Checkmark} & \textcolor{OliveGreen}{\Checkmark} & \textcolor{OliveGreen}{\Checkmark} & \textcolor{Red}{\XSolid} \\
        \hline
        \textbf{Násobné zpomalení Regexeru} & 49x & 29x & 34x & 320x & 20x & 35x & 45x & 4000x \\
        \hline
    \end{tabular}
	\caption{Výsledky testování výkonu zpracování regulárních výrazů}
	\label{tab:DebuggerUI}
\end{table}

V tabulce \ref{tab:DebuggerUI} se nachází testovací výsledky časů zpracovávání.
\textbf{Regulární výrazy a testované řetězce} vychází z přílohy \ref{sec:TestingData}. 
Každý testovací subjekt je označený číslem, které značí regulární výraz a písmenem, které signalizuje vybraný testovací řetězec z listu pod výrazem.
Nalezený výsledek, značí zda testovací řetězec byl nalezen, či nikoliv. 
Násobné zpomalení \textit{Regexeru}, je porovnání oproti JavaScriptové implementaci \textit{RegExp}.

Z výsledků je patrné, že vlastní implementace je značně pomalejší a to hlavně v případě nenalezeného výsledku.
Zpomalení i ve větší míře je očekávané, jelikož se jedná o implementaci psanou v jazyce TS resp. JS.
Také vlastní implementace, má za úkol vytvářet strukturu historie zpracování, což standardní implementace neřeší.
Velké zpomalení při neúspěchu mohlo nastat tak, že každé zpracování probíhá pomocí NKA, ale dnešní jazyky často implementují hybridní variantu DKA s NKA.
DKA v tomto případě má velkou výhodu, jelikož se nemusí vykonávat tolik operací.

Výkonově je tedy vlastní implementace \textit{Regexer} pomalejší, ale pro účely této aplikace je tento výkon dostačující.
Pro zlepšení výkonu, by bylo možné přepsat tuto část aplikace do více nízkoúrovňového jazyka, jako je například \textit{C++}, nebo \textit{Rust}.


% Seznam literatury
\printbibliography[title={Literatura}, heading=bibintoc]

% Prilohy
% TODO: appendix
\appendix

\chapter{Adresářová struktura projektu}\label{sec:FileStructure}

\dirtree{%
.1 /\DTcomment{Kořenový adresář projektu}.
.2 libraries\DTcomment{Adresář vlastních knihoven}.
.3 regex-visualization\DTcomment{Vizualizační knihovna}.
.4 customTypes\DTcomment{Vlastní typy}.
.4 src\DTcomment{Adresář zdrojových kódů}.
.5 assets\DTcomment{Zdroje např. obrázky}.
.5 core\DTcomment{Logická část knihovny}.
.5 styles\DTcomment{Styly aplikace}.
.5 templates\DTcomment{HTML vzory}.
.3 regex-visualizer-extension\DTcomment{Knihovna rozšíření do VSCode}.
.4 release\DTcomment{Výsledné vydání aplikace}.
.4 src\DTcomment{Adresář zdrojových kódů}.
.5 handlers\DTcomment{VSCode handelers}.
.5 providers\DTcomment{VSCode providers}.
.5 web\DTcomment{Adresář pro práci s webview}.
.3 regexer\DTcomment{Knihovna pro zpracování regulárních výrazů}.
.4 \_\_tests\_\_\DTcomment{Kódy pro testování}.
.4 src\DTcomment{Adresář zdrojových kódů}.
.5 core\DTcomment{Logická část knihovny}.
.5 coreTypes\DTcomment{Vlastní typy}.
.5 exceptions\DTcomment{Výjimky}.
.5 structures\DTcomment{Různé stuktury}.
}

% Priloha vlozena primo do hlavniho LaTeX souboru. Ne vsechny prilohy je nutne mit ve zvlastnich souborech.
% \chapter{Dlouhý zdrojový kód}

\end{document}
